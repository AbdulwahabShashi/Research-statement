\documentclass[12pt]{article}
\usepackage[utf8]{inputenc}
\usepackage[small]{titlesec}
\usepackage{fullpage}
\usepackage{amsmath,amsthm}
\usepackage{amssymb}
\usepackage{mathrsfs}
\usepackage{bbm}
\usepackage{comment}
\usepackage{subfiles}
\usepackage{enumitem}
\usepackage{graphicx}
\usepackage{color}
\usepackage{xcolor}
\usepackage{tcolorbox}
\usepackage{float}
\usepackage{xr-hyper}
\usepackage[hidelinks]{hyperref}
\usepackage{cleveref}
\numberwithin{equation}{section}
\usepackage{float}


%\usepackage{bibtex}
%\usepackage[
 %   backend=biber,
 %   style=alphabetic,
 %   maxnames=50,
 %   firstinits=false,
 % ]{biblatex}

%\addbibresource{references.bib}

\newtheorem{theorem}{Theorem}[section]
\newtheorem*{theorem*}{Theorem}
\newtheorem{corollary}[theorem]{Corollary}
\newtheorem{lemma}[theorem]{Lemma}
\newtheorem{proposition}[theorem]{Proposition}
\theoremstyle{definition}
\newtheorem{definition}[theorem]{Definition}
\newtheorem{example}[theorem]{Example}
\newtheorem{assumption}[theorem]{Assumption}

\theoremstyle{remark}
\newtheorem{remark}[theorem]{Remark}
\definecolor{brown}{rgb}{0.5, 0.21, 0.15}

\definecolor{darkgreen}{rgb}{0.05, 0.7, 0.06}
\newcommand{\brown}[1]{\textcolor{brown}{#1}}
\newcommand{\PM}{\mathbb{P}}
\newcommand{\E}{\mathbb{E}} % \E abbreviation for expectation
\newcommand{\Log}{\operatorname{Log}}
\newcommand{\hol}{\mathrm{hol}}
\newcommand{\Ad}{\mathrm{Ad}}
\newcommand{\id}{\mathrm{id}}
\newcommand{\dif}{\,\mathrm{d}}
\newcommand{\diff}{\mathrm{d}}
\newcommand{\R}{\mathbb R}
\newcommand{\Law}{\mathrm{Law}}
\newcommand{\1}{\mathbf 1}
\newcommand{\<}{\langle}  
\renewcommand{\>}{\rangle}
\definecolor{blueblue}{rgb}{0.3, 0.3, 0.95}
\newcommand{\red}[1]{\textcolor{red}{#1}}
\newcommand{\blue}[1]{\textcolor{blueblue}{#1}}
\newcommand{\green}[1]{\textcolor{darkgreen}{#1}}
\newcommand{\orange}[1]{\textcolor{orange}{#1}}

\newcommand{\bfA}{\mathbf A}
\newcommand{\bA}{\mathbb A}
\newcommand{\bfOmega}{\boldsymbol{\Omega}}

\newcommand{\cA}{\mathcal A}
\newcommand{\cB}{\mathcal B}
\newcommand{\cD}{\mathcal D}
\newcommand{\cF}{\mathcal F}
\newcommand{\cG}{\mathcal G}
\newcommand{\cH}{\mathcal H}
\newcommand{\cI}{\mathcal I}
\newcommand{\cJ}{\mathcal J}
\newcommand{\cK}{\mathcal K}
\newcommand{\cL}{\mathcal L}
\newcommand{\cP}{\mathcal P}
\newcommand{\cR}{\mathcal R}
\newcommand{\cS}{\mathcal S}
\newcommand{\cT}{\mathcal T}
\newcommand{\cU}{\mathcal U}
\newcommand{\cX}{\mathcal X}
\newcommand{\cZ}{\mathcal Z}

\newcommand{\bHa}{\mathbb{H}_{\mathsf{a}}}
\newcommand{\bHc}{\mathbb{H}_{\mathsf{c}}}

\newcommand{\rmD}{\mathrm{D}}

\newcommand{\fG}{\mathfrak{G}}

\newcommand{\bC}{\mathbb C}
\newcommand{\T}{\mathbb T}
\newcommand{\bH}{\mathbb H}
\newcommand{\bN}{\mathbb N}
\newcommand{\bP}{\mathbb P}
\newcommand{\bQ}{\mathbb Q}
\newcommand{\bS}{\mathbb S}
\newcommand{\bT}{\mathbb T}
\newcommand{\bZ}{\mathbb Z}
\newcommand*\recvert[1]{\left[\!\left]#1\right[\!\right]}

\newcommand{\vertiii}[1]{{\left\vert\kern-0.4ex\left\vert\kern-0.4ex\left\vert #1 
    \right\vert\kern-0.4ex\right\vert\kern-0.4ex\right\vert}}




    
\title{Research statement}
\author{Abdulwahab Mohamed}
%\author{***}


\begin{document}

\maketitle


\section{Introduction}
My broad field of study is stochastic analysis, with a particular focus on singular SPDEs, constructive quantum field theories (QFTs), and rough analysis. I am also interested in  areas such as stochastic control and regularization by noise.
%
My current research focuses on quantum gauge theory, in particular two and three dimensional Yang-Mills (YM) theories. The Yang-Mills theories, or broadly,   QFTs are fundamental to our understanding of the physical universe. They form the basis of the Standard Model of particle physics. Quantum gauge theories, a class of QFTs, have an important feature, namely gauge symmetries which basically describe how particles interact or change under coordinate transformation. 

On the mathematical side, quantum gauge theories have inspired the development of sophisticated mathematical frameworks and techniques. In particular, my research is based on the study of the YM measure. To explain further, fix a principal $G$-bundle with a base smooth manifold $M$ and a compact structure Lie group $G$. On this principal bundle, which we simply assume to be trivial, one can obtain connection 1-forms $A\in \Omega^1(M,\mathfrak g)$ (where  $\mathfrak g$ is the Lie algebra of $G$).  The YM measure  is formally given by the ill-defined expresssion
\begin{align}\label{eq:YM_measure}
\mu_{\mathrm{YM}}(\diff A)=\frac 1 Z\exp\left(-\int_{M}|F^A(x)|_{\mathfrak g}^2\,\diff x\right)\diff A,
\end{align}
where $F^A$ is the curvature $2$-form of $A$. Making sense of this expression is a central problem in (probabilistic) QFTs. 

There are many works in defining and studying the measure, most interesting works are limited to two—and occasionally three—dimension. The physically relevant case, however, lies in four dimensions, which due to its relevance and difficulty, leading to the Millennium Prize Problem known as the Yang-Mills existence and mass gap problem. As the list of references is long, I only mention the ones that are relevant for my research.  For example, the works \cite{Driver89} and \cite{GKS89} study the measure in the case $M=\R^2$ and exploit the topological property of $\R^2$ to relate the measure to some Gaussian process. The parallel transport of the connection form sampled from the measure, also called the holonomy, solve stochastic differential equations. 
%
%That yields the study of holonomies as Lie group valued Brownian motions.
%
The relation between axial gauge and the so-called lasso fields from \cite{Driver89} is crucial in obtaining a gauge fixing procedure for the 2D Yang-Mills measure on the unit square (see \Cref{sec:RUC_square}). The construction of the measure on generic smooth surfaces $M$ is done in  \cite{Sengupta97}, and later with a graph approximation in \cite{Levy03}. 

Most of these works consider the holonomy process and the connection form in the so-called axial gauge (e.g.\ in the cases of $\R^2$ or $[0,1]^2$). The regularity for the connection form in this case, at least if one could obtain one from the holonomy, has a Holder-Besov regularity of $C^{-1/2^-}$. By a suitable gauge fixing procedure in \cite{Chevyrev19} it is shown that one can actually obtain a connection form in $C^{0^-}$. This was a result, or rather a corollary from a rough Uhlenbeck compactness on $\bT^2$ proved in \cite{Chevyrev19}. The result therein uses lattice approximation of the measure. One of the extensions that I worked on during my PhD is to reprove the result without using lattices, but instead rely on singular SPDE techniques directly in the continuum (see  \Cref{sec:RUC_square}). Although my approach to using singular SPDEs differs slightly, significant progress in studying the YM measure via singular SPDEs—particularly through the Langevin dynamic—has also been made in the incomplete list of works such as  \cite{CCHS2d,CCHS3d,CH23,BC23,BC24}. Techniques from the theory of regularity structures \cite{Hairer14} and paracontrolled calculus \cite{GIP15}, as used in the aforementioned works, are also central to my research. 

Another direction that one could study the measure, is through operator approach. This provides an alternative framework which also helps us understand the underlying geometry better. It is known in classical gauge theory that a connection 1-form $A$ can be equivalently be understood as a covariant derivative $\diff_A$. That is the second focus for my PhD thesis. We try to generalise well-known ideas from the smooth setting to cases where the connection form is distributional, particularly $A\in C^{0^-}$ in 2D, and in the future extend the results to $A\in C^{-1/2^-}$ in 3D. More precisely, we study the operator $\diff_A\oplus\diff_A^*$ and its properties as an unbounded operator on a suitable Banach space (see \Cref{sec:Dirac_2D}). 

I am planning to work on problems related to or arising from my PhD research, for example generalising these results to different geometries or different dimensions to deepen our understanding of YM theories (see \Cref{sec:future} for more details).  At the same time, I am open to exploring other directions within stochastic analysis that may diverge from or build upon my current work.

\section{PhD Research Overview}
%PhD Thesis title: 
%\begin{itemize}
 %   \item Rough Connections
  %  \item Rough Uhlenbeck compactness on compact surfaces
  %  \item  Towards Two-dimensional Rough Gauge Theory and Application to Yang-Mills theory
  %  \item Rough connections, Rough Uhlenbeck Compactness, and application to 2D Yang-Mills measure 
%\end{itemize}

%\subsection{PhD research projects}

\subsection{Rough Uhlenbeck compactness on the unit square}\label{sec:RUC_square}
\textit{This is a joint work in progress with Ilya Chevyrev (University of Edinburgh/TU Berlin) and Tom Klose (University of Oxford).}

\medskip

\noindent In this work we generalise the result in \cite{Chevyrev19} using continuum PDE techniques inspired by the original paper \cite{Uhlenbeck82}. As explained in the introduction, we have the notion of a connection form $A\in\Omega^1(M,\mathfrak g)$. There is a gauge group acting on it by (sufficiently regular) functions $g:M\to G$ via $A^g=gAg-\diff gg^{-1}$. We denote this relation by $\sim$, i.e.\ $A\sim B$ if there is such $g$ such that $A^g=B$.  The Uhlenbeck compactness theorem essentially says that under smallness assumption one can gauge transform a given connection form $A$ to $A^g$ such that 
\[
\|A^g\|_{W^{1,p}}\lesssim \|F^A\|_{L^p}.
\]
Crucial is that the $L^p$-norm of the curvature is gauge-invariant, i.e.\ $\|F^{A^g}\|_{L^p}=\|F^A\|_{L^p}$. In the setting of 2D Yang-Mills theory, the curvature has the same regularity as white noise which makes the $L^p$-norm not applicable. Furthermore, generally Besov norms of the curvature are not gauge invariant.

Therefore, we need to come up with a new quantity. Let us now consider the $M=[0,1]^2$. In this case, we can consider so-called Lasso fields, which are somehow related to the curvature by $L^A=g_AF^Ag_A^{-1}$ for suitable chosen $g_A:M\to G$. These satisfy 
\[
L^{A^g}=g(0)L^Ag(0)^{-1},
\]
which makes the lasso fields  gauge invariant up to a trivial action $G$. 

Recall that connection forms on $M=[0,1]^2$ have two components, i.e.\ $A=A_1\diff x^1+A_2\diff x^2$. A connection form $A$ is said to be in the axial gauge, if $A_2\equiv 0$ and $A_1(\cdot,0)=0$. One important property of the lasso field is that it yields to any $A$ an axial gauge representative $\bar A\sim A$ and it is given explicitly by 
\[
\bar A_1(x)=\int^{x_2}_0 L^A(x_1,t)\dif t, \ \ \  \bar A_2(x)=0. 
\]
This observation allows us to translate the task looking for norms for $L^A$ to $\bar A$, i.e.\ the unique (up to an action of $G$) representative of $A$. 

Moreover, the axial gauge representative of the Yang-Mills measure has very nice statistical properties that allows for integration along line segments. Therefore, we define for the space of axial gauge a space $\bfOmega$ containing elements $\bfA=(A,\bA)$ where $A$ is the integrated connection form and $\bA$ essentially the rough path lift of it. 


I defined a (distributional) space $\bfOmega$ consisting of $\bfA=(A,\bA)$, where $A$ is an integrated 1-form and  $\bA$ its ``rough path" enhancement. 
%
%Smooth $1$-forms are in $\bfOmega$, $A$ is simply the line integral and $\bA$ the line integral against itself. 
%
This is novel and natural as line integration of 1-forms is a natural operation and $\bfOmega$ is a level-2 rough path extension of it. 
%
One has morally $C^\infty\subset \bfOmega\subset C^{-1/2-\kappa}$ and that $\bfOmega$ has a natural notion of gauge transformation. 
%have studied properties of this space with the theory of rough paths and extended the notion of gauge transformation via controlled rough path theory. 

%\textbf{Coulomb condition + Boundary + identification}
We got inspired from the classical theory and to any $\bfA\in\bfOmega$,  we try to find $B$ gauge equivalent to $A$ such that $\diff^*B=0$ over a small square $\Lambda^\sigma=[0,\sigma]^2$. From there, I have derived a suitable system of singular SPDEs with non-trivial boundary conditions which I then solved with regularity structures (modulo adaptations). 
%
%\textbf{Regularity structures + Picard iteration}
%
The singular SPDE yields singular objects 
that one needs to make sense of for which normally one would use probabilistic bounds. I have managed to define the singular objects from $\bfA\in\bfOmega$ pathwise. This is surprising, as the singular objects are not only more numerous but also qualitatively distinct from the information contained in the much simpler, geometrically more natural, quantity $\bfA=(A,\bA)$.  This task is challenging and it is highly non-standard in regularity structure. For example, it requires solution spaces  with certain symmetries, which leads to adaptations to the usual fixed point map to preserve these symmetries. Without those symmetries, there could be an even larger set of singular objects which we cannot define from $\bfA$. 
%
To actually define the singular objects from $\bfA$, I derived analytic identities relating convolution of Green function with line integrals (and iterated line integrals), derivative and integration identities, rough paths techniques, as well as systematic way of treating the singular objects. 

Finally, I have shown regularity for $B$, slightly stronger than $C^{-\kappa}$. Afterwards, one needs to glue solutions from smaller squares to obtain a global connection form on $\Lambda$. I generalized the gluing technique from Uhlenbeck's [1982] to the distributional setting (morally $C^{-\kappa}$ connection forms).



\subsection{Operator approach to 2D Yang-Mills theory via Dirac operators}\label{sec:Dirac_2D}
%
\textit{This is joint work with Ilya Chevyrev (University of Edinbrugh/TU Berlin) and Massimiliano Gubinelli (University of Oxford).} 

\medskip 

\noindent Let $G\subset U(N)$ be a compact Lie group. Let us consider a trivial principal $G$-bundle with base manifold $\bT^2$. We can consider an associated vector bundle via a representation $G\to \operatorname{GL}(\mathbb C^N)$. It is known that connection forms $A$ can be viewed as a covariant derivative on a suitable associated vector bundle. In this project, we study properties of the space of connection forms via using covariant derivative $\diff_A$, or rather more precisely, the operator  $\mathrm{D}_A:=\diff_A\oplus\diff_A^*$ acting on the exterior algebra $\Omega=\bigoplus_{k=0}^ 2\Omega^k(\bT^2,\mathbb C^N)$. 

For a Banach space $X$ of functions we denote by $\Omega X$ the space of differential forms constitued by functions in $X$. We view the operator $\rmD_A$ as an unbounded operator on $\Omega L^2$. We define a suitable domain for $\rmD_A$ to make it a closed, self-adjoint operator with compact resolvent. We can show that the spectrum consists of discrete set eigenvalues with finite multiplicities. We can use suitable resolvent estimates and expansions to show that the orbit space $\Omega C^{\alpha-1}/\mathfrak G^\alpha$ is Polish.

Furthermore, we can establish a natural gauge invariant observables via spectral properties  in the case of $G=U(N)$ and recover gauge transformations via these observables in this case.  



\section{Future Research Directions}\label{sec:future}
In the future, I am planning to extend the results of my PhD thesis to more sophisticated settings. Broadly speaking, I am always open to explore directions in rough analysis that are not necessarily mentioned in this list. 

%The general problems I would like to work on are those in intersection of SPDEs, rough analysis, functional analysis and geometry. 

%General explanation, functional analysis, probability interaction, spdes, pdes, finding more structures, generalising ideas, global existence 

\subsection{Rough Uhlenbeck compactness on closed surfaces}
%
This project builds on the rough Uhlenbeck compactness in my PhD thesis, extending them to a generic closed smooth surface $M$ as the base manifold of the trivial principal $G$-bundle. I have been developing this in parallel, with the possibility of including preliminary findings in my thesis.
%
%This project is generalising the previous work Rough Uhlenbeck in my PhD thesis, but instead consider a generic closed smooth surface $M$ as base manifold. I have been working on this in parallel and it could possibly appear in my PhD thesis.
%

As in the classical setting, the main work involves proving the existence of the Coulomb gauge on a small Euclidean ball; in our setting, however, it suffices to consider a small Euclidean square. We need to change the definition of the metric for of the connection forms slightly. The idea is to consider a suitable graph of the manifold and consider the supremum of the metric on each face (as the faces are homeomorphic to squares). From here onwards we apply similar techniques as explained in \Cref{sec:RUC_square}. 


%We have to make sure that each face of the graph is small enough (in fact smaller than the injectivity radius). On each face, we use the techniques developed in the PhD thesis. We consider a further a subgraph on each face. Each subfaces is then mapped to a Euclidean square on which we have the smallness assumptions to find a Coulomb gauge as done in the rough Uhlenbeck compactness on the unit square. 

The main difficulty in applying this to Yang-Mills theory lies in the fact that, unlike the square case, the lasso field—or effectively, the curvature—does not behave as white noise. Instead, we work with a conditional measure on the initial graph of the manifold. On each face, we introduce a lasso field and derive bounds based on the conditional measure, which, while not strictly Gaussian, possesses comparable Gaussian bounds.
%
\subsection{Several extensions of covariant approach to Yang-Mills theory}\label{sec:ext_Dirac}
%
There are several extensions of the covariant approach, we mention three: (1) discrete approximation of covariant derivatives and new definition of the discrete Yang-Mills measure, (2) extension to compact surfaces, and (3) extension to three dimensional manifolds.  These are all natural next steps of this approach and it is part of the broader program. 

We comment mostly on (1) and (3). The point (1) is an innovative direction towards defining the Yang-Mills measure. It would require new techniques in defining the YM measure. It could also lead to enhancing our understanding of the 3D case which until now is open which brings us to point (3). Extension to three dimensions would provide an alternative state space for the Yang-Mills Langevin dynamic as treated in \cite{CCHS3d} (of course as well as the measure). Combining the theory of covariant approach, we could strengthen some of the properties of the orbit space of the Langevin dynamic, e.g.\ proving Polishness which is still open.   


\subsection{Covariant GFF and constructing conditional Higgs measure}
With Dr.\ Ajay Chandra (Imperial College London), we are planning to study conditional Higgs measure, by first studying a covariant GFF. That is, for a given (rough) $A$, we would like to study the Gibbs-type measure
\[
\mu_A(\diff\Phi)=\frac 1 {Z_A} \exp\left( -\int_{\bT^2}|\diff_A\Phi(x)|^2+c_A|\Phi(x)|^2\diff x\right)\diff\Phi. 
\]
Of course, the way it is written, this measure is ill-defined. This measure is representing a Gaussian free field with covariance kernel given by the Green function for $(\diff_A^*\diff_A+c_A)$. There are several ideas that one could try: study the measure via the covariant Laplacian, Dirichlet form or through the Feynman-Kac formulation of the covariance kernel from e.g.\ \cite{SP24}. The renormalised Laplacian in the Abelian case is constructed in \cite{MM22}. The authors in there use $A$ being Gaussian free field, while in our case we want to sample from the gauge field marginal under the Yang-Mills measure which is slightly different. Important work in the study of gauge field marginal is \cite{CC24}. It would be important to extend such result to non-Abelian setting for the general case.   

The goal is to construct this measure in the Abelian case and later extend it to non-Abelian setting. The way to rigorously construct this measure is by first taking a mollified/discrete $A^\varepsilon\to A$ and consider the measure $\mu_{A^\varepsilon}$ and show convergence (where we need to take $c_{A^\varepsilon}\to-\infty$). Once we have this measure we explore directions to construct the conditional Yang-Mills-Higgs measure.  

%\subsection{Alternative proof for rough Uhlenbeck compactness in 2D}

%Natural quantities, ... Langevin dynamic for example, from Dirac operators?, also helpful for 3D if one finds a suitable way, global existence for pure 2D YM etc, we use the structure of 2D, Broux-Otto-Steele techniques, or only Otto-et al ``energy'' technique


%Alternative proof for rough Uhlenbeck compactness is not for the sake of the insane amount of possibilies to reporve the same result. It will teach us more on  singular SPDEs techniques,  rough differential geometry,and gauge invariant quantities.

%The motivation for developing an alternative proof of rough Uhlenbeck compactness is not merely to replicate the result through numerous methods. Instead, this enhances our understanding of singular SPDE techniques, rough differential geometry, and gauge-invariant quantities. For example, one could try to prove using techniques relying on implicit function theorem as classically done in \cite{Uhlenbeck82} which is not currently available for singular SPDEs.
%
%(even though there is someone working on such thing \orange{Ref?}). 
%
%In fact, in the project \Cref{sec:RUC_square} from my PhD thesis, we developed several interesting techniques to reach this result, and I believe that redoing the proof could reveal even more valuable insights.

%The gauge-invariant observable that we use is unfortunately not easily applicable for the Langevin dynamic. Changing the observable to a different quantity and reporving the same result could be more natural for the Langevin dynamic. One example could be the $L^p$-norm of the curvature smoothed by the Yang-Mills heat flow with some blow-up controlled suitably in the time parameter. Another one could be spectral quantities that can be obtained from Dirac operator $\rmD_A$, for example trace of the heat kernel. 

%Further benefits for tweaking the result to accommodate the 2D Langeivn dynamic is that it could initiate proving pathwise global existence of the orbits under the dynamic as well as global existence for the non-Abelian Yang-Mills-Higgs. Furthermore,  it could enhance our understanding of the 3D Yang-Mills measure. For instance, the result that we currently have in \Cref{sec:RUC_square} using the axial gauge relation to lasso field which is very special property of 2 dimensions (in fact, we know even on $\bT^2$, we do not have such relation).  


%\subsection{Extending regularity structures for boundary singularities}
%In this project I am planning to extend some definitions in regularity structures \cite{Hairer14} to allow for singular objects arising because of boundary considerations. The main example that I have in mind is parabolic SPDEs where one cannot classically close the fixed point equation in the space of modelled distributions. This has occurred in \cite{CCHS3d} for example where the treatment was an additional Da-Prato Debussche step. Generally, if one has situations where one has a large amount of elements that require to be dealt with manually, say if one would consider a Langevin dynamic in a similar framework as  \cite{CM24}, then one needs a unifying technique to deal with such problems. 

%Furthermore, inspiration arising from equations with spatial boundary, where the boundary values has some noise which has some correlations with interior noise. This has occured in the project of \Cref{sec:RUC_square}, where we had a particular technique to deal with. Later this could require a special version of \cite{CH16} to deal with these stochastic objects which could be a natural follow-up project. 

%\subsection{Exploring rough Calderon problems}
%This is a more exploratory project. Calderon type problems, or inverse problem already arises in the study of the operator $\rmD_A=\diff_A\oplus\diff_A^*$ from \Cref{sec:Dirac_2D}. One could try to obtain information of $A$ by studying $\rmD_A$, as well as stability results, i.e.\ some kind of continuity of the map $\rmD_A\mapsto A$. There is many results for operators of the form $\Delta+\zeta$ as well as recently for covariant Laplacian $\diff_A^*\diff_A$ and characterising up to gauge transformation \cite{Cekic20}. These questions are not precisely the same as the one I have mentioned, but it regardless poses the question: what happens when $\zeta$ is a 2D white noise, or $A$ is Gaussian free field on $\mathbb T^2$? Of course, there are issues: firstly the corresponding equations require singular SPDE technique, secondly boundary conditions, specifically, the Dirichlet-to-Neumann map, might require special effort to be understood. 




%\subsection{Regularity structures incorporating initial data issues}
%Also what about spatial boundaries? These could also cause similar issues, how to handle? We have a similar situation in RUC. 







%\subsection{Revisiting Unsolved Problems}
%During the first period of my PhD I have worked on several interesting problems that did not go anywhere. There are two of them which I want to revisit. 

%Lasso fields played an important role in RUC. One could ask whether lasso fields are well-defined as distribution for elements in $C^{0^-}$ (namely the same regularity as GFF). For instance, it is not clear for the YM Langevin dynamic whether there is a way of making sense of the lasso fields in a pathwise manner. The question here, whether one can find a topology for which the GFF lives in, and one can canonically define the lasso field. What we know from pathwise techniques, is that one should aim to find an expansion in terms of stochastic objects, think of modelled distributions or germs. The issue that we have previously faced was that the expansion of the lasso field has functions of the lasso field in the expansion. Alternatively, one can write an equation in terms of two dimensional rough integration where the ``noise'' depends on the solution. Of course this is a bad sign. 

%The new idea is to really use the fact that one already knows the lasso field in the smooth setting. One could hope for the smooth approximation to converge via smallness assumptions of the equantities involved. 

\subsection{Yang-Mills via curvature description}
The discrete Yang-Mills measure is defined through holonomy processes \cite{Levy03}. We know that the holonomy, connection form, and covariant derivative are all equivalent description of the same object (at least in the smooth setting). We already mentioned the discrete approximation in \Cref{sec:ext_Dirac}, so one might wonder how the discrete approximation in terms of the connection form would look like. That is the goal of this project. We consider a finite dimensional approximation of connection forms with finite dimensional gauge group acting on them such that the Yang-Mills measure as written in  \eqref{eq:YM_measure} makes sense on a quotient space. The idea is to prove that such measure converges as the approximation gets finer. 

The new methodology is based on the new enhanced understanding of the spaces on where the connection form lives in due to the works \cite{Chevyrev19,CCHS2d} as well as the extension I have in \Cref{sec:RUC_square}, and rough Uhlenbeck compactness. 


%\section{Broader Impact and Applications}
\bibliographystyle{alpha}
\bibliography{references}


\end{document}


