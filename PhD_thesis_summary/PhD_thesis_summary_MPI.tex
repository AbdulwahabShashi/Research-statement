\documentclass[12pt]{article}
\usepackage[utf8]{inputenc}
\usepackage[small]{titlesec}
\usepackage{fullpage}
\usepackage{amsmath,amsthm}
\usepackage{amssymb}
\usepackage{mathrsfs}
\usepackage{bbm}
\usepackage{comment}
\usepackage{subfiles}
\usepackage{enumitem}
\usepackage{graphicx}
\usepackage{color}
\usepackage{xcolor}
\usepackage{tcolorbox}
\usepackage{float}
\usepackage{xr-hyper}
\usepackage[hidelinks]{hyperref}
\usepackage{cleveref}
\numberwithin{equation}{section}
\usepackage{float}
\usepackage{extpfeil}


%\usepackage{bibtex}
%\usepackage[
 %   backend=biber,
 %   style=alphabetic,
 %   maxnames=50,
 %   firstinits=false,
 % ]{biblatex}

%\addbibresource{references.bib}

\newtheorem{theorem}{Theorem}[section]
\newtheorem*{theorem*}{Theorem}
\newtheorem{corollary}[theorem]{Corollary}
\newtheorem{lemma}[theorem]{Lemma}
\newtheorem{proposition}[theorem]{Proposition}
\theoremstyle{definition}
\newtheorem{definition}[theorem]{Definition}
\newtheorem{example}[theorem]{Example}
\newtheorem{assumption}[theorem]{Assumption}

\theoremstyle{remark}
\newtheorem{remark}[theorem]{Remark}
\definecolor{brown}{rgb}{0.5, 0.21, 0.15}

\definecolor{darkgreen}{rgb}{0.05, 0.7, 0.06}
\newcommand{\brown}[1]{\textcolor{brown}{#1}}
\newcommand{\PM}{\mathbb{P}}
\newcommand{\E}{\mathbb{E}} % \E abbreviation for expectation
\newcommand{\Log}{\operatorname{Log}}
\newcommand{\hol}{\mathrm{hol}}
\newcommand{\Ad}{\mathrm{Ad}}
\newcommand{\id}{\mathrm{id}}
\newcommand{\dif}{\,\mathrm{d}}
\newcommand{\diff}{\mathrm{d}}
\newcommand{\R}{\mathbb R}
\newcommand{\Law}{\mathrm{Law}}
\newcommand{\1}{\mathbf 1}
\newcommand{\<}{\langle}  
\renewcommand{\>}{\rangle}
\newcommand{\sign}{\operatorname{sgn}}
\newcommand{\tr}{\operatorname{tr}}


\newcommand{\hor}{\text{-}\mathrm{hor}}
\newcommand{\hgr}{\text{-}\mathrm{hgr}}
\newcommand{\ax}{\text{-}\mathrm{ax}}
\newcommand{\gr}{\text{-}\mathrm{gr}}


\newcommand{\rdom}{{\hspace{1pt}\scalebox{0.8}{\raisebox{0.65pt}{$\blacktriangleleft$}}\hspace{1pt}}}
\newcommand{\ldom}{{\hspace{1pt}\scalebox{0.8}{\raisebox{0.65pt}{$\blacktriangleright$}}\hspace{1pt}}}

\newcommand{\res}{\hspace{-1.2pt}\bullet\hspace{-1.2pt}}


\definecolor{blueblue}{rgb}{0.3, 0.3, 0.95}
\newcommand{\red}[1]{\textcolor{red}{#1}}
\newcommand{\blue}[1]{\textcolor{blueblue}{#1}}
\newcommand{\green}[1]{\textcolor{darkgreen}{#1}}
\newcommand{\orange}[1]{\textcolor{orange}{#1}}

\newcommand{\bfA}{\mathbf A}
\newcommand{\bA}{\mathbb A}
\newcommand{\bfi}{\mathbf i}
\newcommand{\bfOmega}{\boldsymbol{\Omega}}

\newcommand{\cA}{\mathcal A}
\newcommand{\cB}{\mathcal B}
\newcommand{\cD}{\mathcal D}
\newcommand{\cF}{\mathcal F}
\newcommand{\cG}{\mathcal G}
\newcommand{\cH}{\mathcal H}
\newcommand{\cI}{\mathcal I}
\newcommand{\cJ}{\mathcal J}
\newcommand{\cK}{\mathcal K}
\newcommand{\cL}{\mathcal L}
\newcommand{\cP}{\mathcal P}
\newcommand{\cR}{\mathcal R}
\newcommand{\cS}{\mathcal S}
\newcommand{\cT}{\mathcal T}
\newcommand{\cU}{\mathcal U}
\newcommand{\cX}{\mathcal X}
\newcommand{\cZ}{\mathcal Z}

\newcommand{\sfZ}{\mathsf Z}

\newcommand{\bHa}{\mathbb{H}_{\mathsf{a}}}
\newcommand{\bHc}{\mathbb{H}_{\mathsf{c}}}

\newcommand{\rmD}{\mathrm{D}}

\newcommand{\fG}{\mathfrak{G}}
\newcommand{\fg}{\mathfrak g}
\newcommand{\fu}{\mathfrak u}

\newcommand{\bC}{\mathbb C}
\newcommand{\T}{\mathbb T}
\newcommand{\bH}{\mathbb H}
\newcommand{\bN}{\mathbb N}
\newcommand{\bP}{\mathbb P}
\newcommand{\bQ}{\mathbb Q}
\newcommand{\bS}{\mathbb S}
\newcommand{\bT}{\mathbb T}
\newcommand{\bZ}{\mathbb Z}
\newcommand{\Div}{\operatorname{Div}}
\newcommand*\recvert[1]{\left[\!\left]#1\right[\!\right]}

\newcommand{\vertiii}[1]{{\left\vert\kern-0.4ex\left\vert\kern-0.4ex\left\vert #1 
    \right\vert\kern-0.4ex\right\vert\kern-0.4ex\right\vert}}
\newcommand{\triple}[1]{\vertiii{#1}}



    
\title{PhD Thesis Summary}
\author{Abdulwahab Mohamed}
\date{}
%\author{***}


\begin{document}

\maketitle
\noindent {\footnotesize \textbf{Expected to finish PhD:} July/August 2025}
\section{Introduction}
My PhD research focuses on the study of two-dimensional Yang-Mills (YM) theory using singular (S)PDE techniques. YM theory plays an important role in both mathematical physics and geometry, and from a probabilistic point of view, much remains to be understood. We are interested in understanding the ill-defined Gibbs-type measure on a manifold $M$ with principal $G$-bundle structure given by
\begin{align}\label{eq:YM_measure}
\mu_{\mathrm{YM}}(\diff A)=\frac 1 Z\exp\left(-\int_{M}|F^A(x)|_{\mathfrak g}^2\,\diff x\right)\diff A,
\end{align}
where $A$ is so-called connection form and $F^A$ is its curvature $2$-form and $\fg$ is the Lie algebra of the Lie group $G$. 
%

This measure is successfully constructed in series of works with different techniques, e.g.\ \cite{Driver89,GKS89,Sengupta97,Levy03}. One crucial difficulty in making sense of the YM measure is that it is defined on space of equivalence classes of essentially connection forms. The equivalence class is defined through gauge transformations.   These works, however, do not allow for gauge-fixing the YM measure to a representative with good regularity. That is first done by my supervisor in \cite{Chevyrev19} via a rough Uhlenbeck compactness result on $M=\bT^2$ using lattice approximation.  

The first aim of my thesis is to prove rough Uhlenbeck compactness using PDE techniques. As the PDE is on the continuum, the challenges that arise are very different than those in \cite{Chevyrev19}. I developed new definitions and techniques to solve the singular SPDE arising in this problem. For example, the result cannot be proved by the black box theory on regularity structures. This is elaborated in \Cref{sec:RUC}. 

The second project is to formulate an operator-based approach to YM theory. We consider the covariant derivative associated with a connection form $A$ and study this operator as an unbounded operator on $L^2$-valued differential forms. This is part of a program to develop an alternative perspective on the theory of YM theory. I detail this project in \Cref{sec:operator}. 

%
Throughout the remaining, we fix  $G\subset U(N)$ to be a compact Lie group and $\fg\subset \fu(N)$ to be its Lie algebra. We consider a trivial principal bundle $M\times G$ where $M=[0,1]^2$ in \Cref{sec:RUC} and $M=\bT^2$ in \Cref{sec:operator}.  




\section{Rough Uhlenbeck compactness}\label{sec:RUC}
\vspace{-7pt}\textit{{\footnotesize Joint work in progress with Ilya Chevyrev (Univ.\ of Edinburgh) and Tom Klose (Univ.\  of Oxford)}}\\

\noindent \vspace{-2pt}Under the setting explained above with $M=\Lambda=[0,1]^2$, we say a connection form $A=A_1\diff x^1+A_2\diff x^2$ is in the axial gauge if $A_2\equiv 0$. There is a suitable space of connection forms in the axial gauge for the YM measure on $\Lambda$, namely $\bfOmega_{\alpha\ax}^1$ with $\alpha\in (\frac 1 3,\frac 1 2)$. This space is inspired by rough paths theory for line integrals (see \Cref{sec:RAF}) and consists of integrated $1$-form $A$ and its enhancement $\bA$. We also recall the space $\Omega_\beta^1$ for $\beta\in (\frac 1 2,1)$ defined in \cite{CCHS2d} which resembles a distributional space where line integration can be carried out under the Young regime. These spaces are quite useful and one morally has $\bfOmega_{\alpha\ax}^1\subset C^{\alpha-1}$ and $\Omega_\beta^1\subset C^{\beta-1}$. 


The main result allows us to gauge fix $1$-forms in $\bfOmega_{\alpha\ax}^1$ to a better behaved one in $\Omega_\beta^1$. It can be formulated as follows: 
\begin{theorem}[Rough Uhlenbeck compactness, unprecise version]\label{thm:RUC} Let $\alpha$ (resp.\ $\beta$) be less than and sufficiently close to $\frac 12$ (resp.\ $1$). Let $\bfA=(A,\bA)\in \bfOmega_{\alpha\ax}$, then there exists gauge transformation $g$ such that $A^g\in \Omega_\beta$ and the map $\bfA\mapsto A^g$ is ``locally Lipschitz" continuous.   
\end{theorem}

 Moreover, the application to the YM measure is then:
\begin{corollary}
    The YM measure on $\Lambda=[0,1]^2$ can be gauge fixed to a $\Omega_\beta^1\subset C^{\beta-1}$ distribution for any $\beta\in (\frac 1 2 ,1)$. 
\end{corollary}


\Cref{thm:RUC}  is proved via singular SPDE techniques and is inspired by the proof of Uhlenbeck compactness in \cite{Uhlenbeck82}. The proof is long and technical. In the remaining subsection, we introduce some notations and ideas of the proof.   
\subsection{Rough additive functions}\label{sec:RAF}

%We start by recalling some basic definitions from \cite{CCHS2d}.  
%
%We fix a finite-dimensional Banach space $E$ and we recall the domain $\Lambda=[0,1]^2$. 
%
We denote by $\cX$ the set of line segments $\ell=(x,v)$ in $\Lambda$, where $x\in\Lambda$ is the starting point and $v\in\R^2$ the direction vector. We denote by $|\ell|$ the length of $\ell$. 
%
We say two lines $\ell,\bar\ell\in\cX$ are \emph{joinable} if $\ell$ ends where $\bar\ell$ starts; in such case, we write $\ell\sqcup\bar\ell$ for the concatenated line. We say that a function $A:\cX\to \fg$ is \emph{additive} if $A(\ell\sqcup\bar\ell)=A(\ell)+A(\bar\ell)$ for all joinable lines $\ell,\bar\ell\in\cX$.  
%
We view $A(\ell)$ as the line integral of a (smooth) connection $1$-form $A $ along $\ell$. Indeed, when we have a smooth $1$-form $A\in\Omega^1 (\Lambda,\fg)$ we set
\[
A(\ell):=\int_\ell A,
\]
with the obvious abuse of notation that $A$ is both the integrated $1$-form and the $1$-form itself. 
%
Finally, for $\beta\in (0,1]$, we recall the norm $|\cdot|_{\beta\gr}$ defined in \cite{CCHS2d}, namely 
\[
|A|_{\beta\gr}:=\sup_{\ell\in\cX,|\ell|\neq 0}\frac{|A(\ell)|}{|\ell|^\beta}.
\]
We define $\Omega^1_{\beta\gr}$ to be the completion of smooth $1$-forms under $|\cdot|_{\beta\gr}$.
%\subsubsection{Definition of rough additive functions}
%
We now define the notion of an \emph{enhanced additive function}, namely 
%
 $\bA:\cX\to \fg\otimes \fg$ which satisfies the following version of Chen's relation: 
%
For all joinable lines $\ell,\bar\ell\in\cX$, we have $\bA(\ell\sqcup \bar\ell)-\bA(\ell)-\bA(\bar\ell)=A(\ell)\otimes A(\bar \ell)$.
We call $\bfA=(A,\bA)$ \emph{enhanced additive function}.  In case $A$ is a smooth connection $1$-form, we set
%
$$\bA(\ell)=\int^1_0\int^r_0 \diff\ell_A(r') \otimes\diff\ell_A(r)=\int^1_0\ell_A(r)\otimes\diff\ell_A(r),$$
%
where the notation $\ell_A:[0,1]\to\fg$ is defined by $\ell_A(t):=A((x,tv))$ for $\ell=(x,v)$. This $\bA$ is reminiscent of the iterated integral in rough paths theory. 
%
Let $\alpha\in (\frac 13,\frac 12)$ and define the following:
\begin{align}
\|\bA\|_{2\alpha\gr}:=\sup_{\ell\in\cX,|\ell|\neq 0}\frac{|\bA(\ell)|}{|\ell|^{2\alpha}}. 
\end{align}
We now let 	\begin{align}\triple{\bfA}_{\alpha\gr}:=|A|_{\alpha\gr}+\|\bA\|_{2\alpha\gr}^{1/2}.\end{align} 
	%
    Note that $\triple{\cdot}_{\alpha\gr}$ induces a natural metric and the closure of smooth $1$-forms under this metric is denoted by $\bfOmega^1_{\alpha\gr}$, also called the space of \emph{rough additive functions}.   

One can define more sohpisticated norms 
 by comparing different line segments as well. In fact, there is a suitable space for axial gauge representatives, as mentioned above, and that space is called $\bfOmega_{\alpha\ax}^1$. One can show via a Kolmogorov continuity theorem that the YM measure under the axial gauge is in  $\bfOmega_{\alpha\ax}^1$ for any $\alpha\in (\frac 1 3,\frac 1 2)$. 

%We naturally have $\bfOmega_{\alpha\gr}\hookrightarrow\Omega_{\alpha\gr}$ and $\bfOmega^1_{\alpha\gr}\hookrightarrow\Omega^1_{\alpha\gr}$. Furthermore, for any $\beta\in (\frac 12,1] $, we have that any $A\in\Omega^1_{\beta\gr}$ can be canonically enhanced by means of Young integration to obtain $\bA$ yielding $\Omega_{\beta\gr}^1\hookrightarrow\bfOmega_{\beta\gr}^1$. Finally, we want to mention that \green{Corollary 3.23} in \cite{Chevyrev18YM} tells us that $\bfOmega^1_{\alpha\gr}\hookrightarrow \Omega C^{\alpha-1}$. 


%\subsubsection{Gauge transformation}%\label{sec:gauge_transformation}
The space of rough additive functions is extremely useful and natural from a geometric point of view. For example, we show how it allows us to define gauge transformations. First recall  $\Ad_g\in L(\fg)$ for a given $g\in G$ is the map $\Ad_gx=gxg^{-1}$.
Let $\bfA=(A,\bA)\in \bfOmega^1_{\alpha\gr}$, and $g:\Lambda\to G$. Define for $\ell\in\cX$ and $\ell_g(t):=g(x+tv)$. For any $\bfA$ we can associate naturally a genuine rough path $\ell_\bfA$.   The formula for gauge transformation $A^g=\Ad_g A-\diff gg^{-1}$ translates to  \begin{align}\label{e:gauge_trafo_connection:2}
		A^g(\ell):=\int^1_0 \Ad_{\ell_g(t)}\dif\ell_{\bfA}(t)-\dif\ell_g(t)\ell_{g^{-1}}(t), \quad \ell \in \cX,
	\end{align}
	%
    which makes sense for any $g$ for which $\ell_g$ is controlled rough path with respect to $\ell_A$. Finally, by rough path theory, one can check that there is a canonical enhancement for $A^g$, denoted by~$\bA^g$. 

   Furthermore, I have defined parallel translation, also called holonomy, via RDEs and proved a method for recovering gauge transformations through holonomies around loops.

\subsection{Derivation and solving an elliptic (S)PDE for Coulomb gauge}
The main ingredient of the proof is inspired by the work of \cite{Uhlenbeck82}, namely to find a Coulomb gauge representative. We recall $\diff^*A=-\sum_{i=1}^2\partial_i A_i$. We say $A$ is in the Coulomb gauge if $\diff^*A=0$. This section derives a system of elliptic (S)PDEs to solve for $\diff^*A^g=0$, for a given $A$ in the axial gauge, in practice the YM measure under the axial gauge.  This task is non-trivial due to the infinite degrees of freedom in formulating PDEs from $\diff^*A^g=0$. Ironically, this flexibility introduces challenges in identifying a solvable equation that yields the result we are after. In our application, we have $A\in C^{-1/2^-}$ as dictated by the regularity of the YM measure under the axial gauge. Moreover, the holonomy characterisation of gauge transformations shows $g\in C^{1/ 2^-}$. By definition $A^g=\Ad_gA-\diff gg^{-1}$, making any SPDE formulation of $\diff^*A^g=0$ singular. Indeed, the product $\Ad_gA$ as well as $\diff gg^{-1}$ do not make sense.  

To set up the equation, we note that $B=A^g$ satisfies $\diff^*B=0$ implying that $B=\diff^*w$ for a $2$-form $w$. Then I have derived the following system of equations for $H=\Ad_g$ and $w$, namely 
  \begin{align}\label{eq:SPDE}
  \begin{split}
          \begin{cases}
          \Delta H&=\partial_1 [HA_1,\cdot]\circ H-\partial_1[\partial_2w,\cdot]\circ H+\partial_2[\partial_1w,\cdot]\circ H\\
          \Delta w&=\partial_2(HA_1)+[\partial_1w,\partial_2w]-[\partial_1w,HA_1].
          \end{cases}
          \end{split}
          \end{align}
I have also derived suitable boundary conditions since it is on smaller squares $[0,\sigma]^2$ to introduce smallness parameter. This is allowed, as we have a patching step to glue solutions from small squares together. The boundary condition on each domain is non-trivial and requires additional care to set up. Finally, I introduced additional manipulations, e.g.\ to deal with boundary regularity issues, %because we are working on a bounded domain, 
which makes \eqref{eq:SPDE} a simplified and naive version of the equation we end up solving.  

I solved the equation using regularity structures where I have done several adaptations to the general theory. One challenge was that the general theory does not preserve the symmetries of the terms. For instance, I defined a non-standard fixed-point map specifically designed to preserve these symmetries.

Finally, there is an identification step, meaning that solving \eqref{eq:SPDE} may not necessarily mean that the final solution $B=\diff^*w$ is indeed $A^g$. More precisely, we need to separately verify that there is a genuine gauge transformation $g$ such that $B=A^g$. I proved this using uniqueness of another linear singular SPDE that $H$ solves.


\subsection{Model construction and input data via rough additive functions}
In the theory of regularity structures, one needs to define a model $\sfZ=(\Pi,\Gamma)$. Our construction is unique as we have 
$$A_1\stackrel{\text{probabilistic}}{\xmapsto{\hspace{1.7cm}}} \textcolor{blue}{\bfA=(A,\bA) \stackrel{\text{deterministic}}{\xmapsto{\hspace{1.7cm}}}} \,\sfZ=(\Pi,\Gamma).$$
Normally, one has to construct from the noise, in our case $A_1$, the model $\sfZ$. In our case, we have the \textcolor{blue}{blue} step in between which allows us to define the model deterministically as well as continuously from a rough additive function $\bfA\in\Omega_{\alpha\ax}^1$. This is surprising as the terms in the regularity structures are more numerous and distinct from the elements in $\bfA$.  I developed and implemented this program independently, and the result was as unexpected to me as it was to us.

This relies on techniques inspired by several heuristics, namely convolution with $K_1=\partial_1 \Delta^{-1}$ on $A_1$ acts as line integration of $A$, and $\partial_2$ reduces regularity by $\frac 1 2+\kappa$ on terms like $A_1$ and more generally on terms that do not contain derivatives with respect to the $x_2$-variable. Finally, some terms can be written as full derivatives. For example
\[
\partial_i f\partial_j g-\partial_j f\partial_i g=\partial_i (f\partial_j g)-\partial_j (f\partial_i g). 
\]
We need additional structure for some terms to enforce them arise as full derivatives. This is one of the reasons we cannot apply the black box theory of regularity structures. Unfortunately, the precise argument is technical and I have developed a semi-systematic way of treating some of the terms.  
%We split the terms in regularity structures in three different type of terms, core, boundary and rest. To define core we use the first two heuristics and for boundary we use more rough paths theory. Finally to define the terms on the rest we have a notion that allows us to reduce the definition of model to core elements. I have developed a systematic framework to deal with the latter. 

There are additional quantities that I had to define manually from $\bfA$, arising due to the low boundary regularity of the modelled distributions. Treating such quantities deterministically is a new phenomenon, which I implemented, guided by rough It\^o formula.


\section{Covariant derivative}\label{sec:operator}
\vspace{-7pt}\textit{{\footnotesize Joint work in progress with Ilya Chevyrev (Univ.\ of Edinburgh) and Massimiliano Gubinelli (Univ.\ of Oxford)}}\\

\noindent \vspace{-2pt}Let the base manifold $M=\bT^2$ and recall the principal $G$-bundle $P$.    To this principal $G$-bundle we associate an associated vector bundle. The bundle is $E=P\times \bC^N/G$, i.e.\ it is a bundle where the fibres look like $\bC^N$. %We can also choose a global section in of $E$ from the global section of $P$.  
Let $A\in\Omega^1(\bT^2,\fg)$ be a connection $1$-form. We want to study the covariant exterior derivative $\diff_A:\Omega^k(\bT^2,\bC^N)\to \Omega^{k+1}(\bT^2,\bC^N)$ and its adjoint $\diff^*_A:\Omega^k(\bT^2,\bC^N)\to \Omega^{k-1}(\bT^2,\bC^N)$. We could add these two maps to obtain the ``Covariant Dirac operator'' $\rmD_A:= \diff_A\oplus\diff_A^*$ as an endomorphism on 
\[
\Omega(\bT^2,\bC^N):=\bigoplus_{k=0}^2 \Omega^k(\bT^2,\bC^N).
\]
We can write $\rmD_A=\rmD_0+\bA$, where $\bA=A\oplus A^*$ (with $A$ really meaning the operation $A\wedge$ and $A^*$ its adjoint). Note that we use $\bA$ for a different object than in \Cref{sec:RUC}. 
From here onwards we avoid writing the spaces $\bT^2$, $\bC^N$ etc, and we simply write $\Omega$ (or $\Omega^k$). 

We study the properties of the covariant Dirac operator $\rmD_A$ as an unbounded operator on $\Omega L^p$. One of the motivations to do so is to use the operator as a way of defining an alternative viewpoint of the YM measure. 


\subsection{Covariant Dirac operator in the Young regime}
Throughout this section we fix $\alpha\in (\frac 1 2,1)$ and a connection form $A\in\Omega^1 C^{\alpha-1}$. Even though $A$ is distributional (as well as $\bA=A\oplus A^*$), we can define an unbounded operator $(\rmD_A,\cD(\rmD_A))$ on $\Omega L^p$. The domain is characterised via paraproducts. I have shown this operator is densely defined, closed and self-adjoint. 

The resolvent $R_A(z):=(\rmD_A-z)^{-1}$ is compact for $z\in \rho(\rmD_A)$ (the resolvent set) and has nice properties. We can also use the resolvent to recover the operator $\bA=A\oplus A^*$:
\begin{lemma}\label{lem:recover_A_Young}
Let $\beta>0$  and $\theta\in [0,\beta)$ such that  $\beta-\theta+\alpha-1>0$, then 
\[
\|(z^2(R_A(z)-R_0(z))+\bA)f\|_{W^{\alpha-1-\theta,p}}\lesssim |z|^{-\theta}\|f\|_{W^{\beta,p}},
\]
along suitable $z$ large enough. In particular as $|z|\to\infty$ we can recover $\bA$ from the limit of $z^2(R_A(z)-R_0(z))$. 
\end{lemma}

We can also define gauge transformations on the level of covariant derivatives, namely $\rmD_A^g=g\rmD_A g^{-1}$ and using the previous lemma, we can show this non-trivial bound:
\begin{lemma}
 Let $\alpha$ sufficiently close to $1$. Then there exists $C>0$ and $m\geq 1$ (depending on $\alpha$) such that 
 \[
 \|g\|_{C^{\alpha}}\leq C(1+\|A\|_{C^{\alpha-1}}+\|A^g\|_{C^{\alpha-1}})^m.
 \]
\end{lemma}
In particular, this is one of the main ingredient that allows to show that the quotient space $\Omega C^{\alpha-1,0}/C^\alpha(\bT^2,G)$ is Polish. In here we use $C^{\alpha,0}$ to mean the closure of smooth functions in $C^\alpha$. This improves upon \cite{CCHS2d} where a stronger norm was required to demonstrate Polishness of the quotient space.  

Finally, I have defined gauge-invariant observables which characterise gauge transformations using spectral properties as well as topological properties and traces of unitary operators and Sobolev embeddings. This result require $G=U(N)$ and smoothening out by YM heat flow. 



\bibliographystyle{alpha}
\bibliography{references}


\end{document}