\documentclass[12pt]{article}
\usepackage[utf8]{inputenc}
\usepackage[small]{titlesec}
\usepackage{fullpage}
\usepackage{amsmath,amsthm}
\usepackage{amssymb}
\usepackage{mathrsfs}
\usepackage{bbm}
\usepackage{comment}
\usepackage{subfiles}
\usepackage{enumitem}
\usepackage{graphicx}
\usepackage{color}
\usepackage{xcolor}
\usepackage{tcolorbox}
\usepackage{float}
\usepackage{xr-hyper}
\usepackage[hidelinks]{hyperref}
\usepackage{cleveref}
\numberwithin{equation}{section}
\usepackage{float}


%\usepackage{bibtex}
%\usepackage[
 %   backend=biber,
 %   style=alphabetic,
 %   maxnames=50,
 %   firstinits=false,
 % ]{biblatex}

%\addbibresource{references.bib}

\newtheorem{theorem}{Theorem}[section]
\newtheorem*{theorem*}{Theorem}
\newtheorem{corollary}[theorem]{Corollary}
\newtheorem{lemma}[theorem]{Lemma}
\newtheorem{proposition}[theorem]{Proposition}
\theoremstyle{definition}
\newtheorem{definition}[theorem]{Definition}
\newtheorem{example}[theorem]{Example}
\newtheorem{assumption}[theorem]{Assumption}

\theoremstyle{remark}
\newtheorem{remark}[theorem]{Remark}
\definecolor{brown}{rgb}{0.5, 0.21, 0.15}

\definecolor{darkgreen}{rgb}{0.05, 0.7, 0.06}
\newcommand{\brown}[1]{\textcolor{brown}{#1}}
\newcommand{\PM}{\mathbb{P}}
\newcommand{\E}{\mathbb{E}} % \E abbreviation for expectation
\newcommand{\Log}{\operatorname{Log}}
\newcommand{\hol}{\mathrm{hol}}
\newcommand{\Ad}{\mathrm{Ad}}
\newcommand{\id}{\mathrm{id}}
\newcommand{\dif}{\,\mathrm{d}}
\newcommand{\diff}{\mathrm{d}}
\newcommand{\R}{\mathbb R}
\newcommand{\Law}{\mathrm{Law}}
\newcommand{\1}{\mathbf 1}
\newcommand{\<}{\langle}  
\renewcommand{\>}{\rangle}
\definecolor{blueblue}{rgb}{0.3, 0.3, 0.95}
\newcommand{\red}[1]{\textcolor{red}{#1}}
\newcommand{\blue}[1]{\textcolor{blueblue}{#1}}
\newcommand{\green}[1]{\textcolor{darkgreen}{#1}}
\newcommand{\orange}[1]{\textcolor{orange}{#1}}

\newcommand{\cA}{\mathcal A}
\newcommand{\cB}{\mathcal B}
\newcommand{\cD}{\mathcal D}
\newcommand{\cF}{\mathcal F}
\newcommand{\cG}{\mathcal G}
\newcommand{\cH}{\mathcal H}
\newcommand{\cI}{\mathcal I}
\newcommand{\cJ}{\mathcal J}
\newcommand{\cK}{\mathcal K}
\newcommand{\cL}{\mathcal L}
\newcommand{\cP}{\mathcal P}
\newcommand{\cR}{\mathcal R}
\newcommand{\cS}{\mathcal S}
\newcommand{\cT}{\mathcal T}
\newcommand{\cU}{\mathcal U}
\newcommand{\cX}{\mathcal X}
\newcommand{\cZ}{\mathcal Z}

\newcommand{\bHa}{\mathbb{H}_{\mathsf{a}}}
\newcommand{\bHc}{\mathbb{H}_{\mathsf{c}}}

\newcommand{\rmD}{\mathrm{D}}

\newcommand{\fG}{\mathfrak{G}}

\newcommand{\bC}{\mathbb C}
\newcommand{\T}{\mathbb T}
\newcommand{\bH}{\mathbb H}
\newcommand{\bN}{\mathbb N}
\newcommand{\bP}{\mathbb P}
\newcommand{\bQ}{\mathbb Q}
\newcommand{\bS}{\mathbb S}
\newcommand{\bT}{\mathbb T}
\newcommand{\bZ}{\mathbb Z}
\newcommand*\recvert[1]{\left[\!\left]#1\right[\!\right]}

\newcommand{\vertiii}[1]{{\left\vert\kern-0.4ex\left\vert\kern-0.4ex\left\vert #1 
    \right\vert\kern-0.4ex\right\vert\kern-0.4ex\right\vert}}




    
\title{Research statement}
\author{Abdulwahab Mohamed}
%\author{***}


\begin{document}

\maketitle


\section{Introduction}
Stuff to mention 
\begin{itemize}
    \item QFT
    \item Mathematical Physics
    \item Millinium problem
    \item Research in singular SPDEs
    \item Understanding of different problems within
    \item Literature here and there
\end{itemize}
\section{PhD Research Overview}
PhD Thesis title: 
\begin{itemize}
    \item Rough Connections
    \item Rough Uhlenbeck compactness on compact surfaces
    \item  Towards Two-dimensional Rough Gauge Theory and Application to Yang-Mills theory
    \item Rough connections, Rough Uhlenbeck Compactness, and application to 2D Yang-Mills measure 
\end{itemize}

%\subsection{PhD research projects}
\subsection{State space for two-dimensional Higgs field and String observables}
In this mini-project we define a state space for the Higgs field $\Phi$ for which string observables are well-defined. We will also show that the string observables can be used to separate the gauge orbits for the Higgs field. 

The context of this mini-project is somewhere in between the two papers \cite{CCHS_2D_YM} and \cite{CCHS_3D_YMH}. While \cite{CCHS_2D_YM} was about stochastic quantisation for the 2D Yang-Mills measure, \cite{CCHS_3D_YMH} was about the stochastic quantisation for the 3D Yang-Mills-Higgs measure. We consider the 2D Yang-Mills-Higgs and all results mentioned could be seen as a word of word translation of the results for the connection 1-forms $A$ to the Higgs fields $\Phi$. In fact, we believe that this mini-project could fit in an hypothetically extended version of \cite{CCHS_2D_YM}.  

However, it turns out even though the state space for the Higgs field is similar, the place of the Wilson loops and holonomies is taken by the string observable. We consider double parameterised curves $\gamma$ of the form $\gamma(s,t)=x+s(v+tw)$ for some vectors $x,v,w\in \R^2$, and we want to make sense of
$$S(A,\Phi,\gamma)=\int^1_0 \hol(A,\gamma_t)\Phi(\gamma_t(1))\,\diff t,$$
where $A$ is the line integral of a connection 1-form along lines and $\hol(A,\gamma_t)$ is the holonomy of such $A$. It is shown in \cite{CCHS_2D_YM} that the holonomy for quite irregular objects is well-defined by means of Young integration. 

We want to apply this in the context of the stochastic quantisation of the Yang-Mills-Higgs in 2D. It turns out that the object $\Phi$ in question is quite irregular, in fact it is not even a function, it is a distribution. As a consequence the point evaluation $\Phi(\gamma_t(1))$ is not well-defined. Fortunately, the issue can be remedied through a very similar construction as in \cite{CCHS_2D_YM} which was done for integrated 1-form $A$.  Indeed, we write
$$S(A,\Phi,\gamma)=\int^1_0 \hol(A,\gamma_t)\,\diff\left( \int^\cdot_0 \Phi(\gamma_s(1))\,\diff s\right)(t),$$
and we exploit Young integration by constructing a space for which the line integral of such $\Phi$ is well-defined. 

After we define the string observable $S$ for a suitable chosen domain, we show that it serves as a separating object for the gauge transformation. We show a statement of the following type:
$$g(x)S(A,\Phi,\gamma)=S(A^g,\bar\Phi,\gamma), \  \ \ \text{ for all doubly affine }\gamma,$$
then $\bar\Phi=\Phi^g$. 

Finally, we generalise the results from Section 3.6 in \cite{CCHS_2D_YM} slightly. We obtain conditions to ensure a suitable quotient space $X/\mathfrak G$ is a Polish space for a Banach space $X$ and topological group $\mathfrak G$. 


\subsection{Rough Uhlenbeck compactness on the unit square}
\textit{This is a joint work in progress with Ilya Chevyrev (UoE) and Tom Klose (UoO).}

\medskip

\noindent In this work we generalise the works Chevyrev 2019 using continuum PDE techniques inspired by the works Uhlenbeck 19??. First we solve for Coulomb gauge $\diff^*A=0$ on smaller squares, and then patch the solutions in each small square together to obtain a globally defined connection form. The latter is very similar to what was also done in Uhlenbeck \red{19??}. Our techniques differs in the sense that we solve for the Coulomb gauge directly, instead of using implicit existence results. 

Classically, the result is essentially under smallness assumptions, one can find a gauge transformation such that 
\[
\|A^g\|_{W^{1,p}}\lesssim \|F^A\|_{L^p}.
\]
In the setting of 2D Yang-Mills theory, the curvature has the same regularity as white noise which makes the $L^p$-norm not applicable. Furthermore, generally Besov norms of the curvature is not gauge invariant, whereas $L^p$-norms are. Instead, we consider so-called Lasso fields, which are somehow related to the curvature. These are gauge invariant up to a trivial action $G$. 

The distribution of the lasso fields $L^A$ for the 2D YM measure on the square is explicit, namely same as white noise. The lasso fields relate to axial gauge representations $\bar A$, in the sense that 
\[
\bar A=\int L^A. 
\]
In our work we consider norms for the axial gauge representative. 

We first develop the theory of rough additive functions. This generalises the notion of additive functions \red{Ref??}. The definition of additive functions to rough additive functions, is how Young integration is to rough integration. There is a natural notion for gauge transformations for such rough additive functions inspired by controlled rough paths. Then our Uhlenbeck compactness boils down whether one can transform a rough additive funciton in the axial gauge to the Coulom gauge. 

The latter yields a singular SPDE. We wish to solve for $g$ such that $\diff^*A^g=0$. Recall that $A^g=\Ad_gA-\diff gg^{-1}=\Ad_gA+0^g$. We set up a system of SPDEs for essentially $\Ad_g$ and $0^g$. As the bound, that we want to prove, does not care about the way we solve for these equations, or whether we solve any equation, this causes a freedom. We use the equations carefully as well with boundary conditions. We use the theory of regularity structures to solve it. However, there is so many adaptations that we have to do from the classical black box theory. 


Firstly, our equation is elliptic. While this per se not that problematic, we have boundary conditions, which cause technical issues which breaks down the theory of regularity structures. This leads us to come up with an smartly chosen auxiliary modelled distribution.  Secondly, we construct the model from a rough additive function (as well as the auxiliary modelled distribution). This requires a heavy machinary as it is a highly non-trivial task. We introduce, suitable sectors for regularity structure and enhanced Picard iteration which preserves these sectors. This is a sophisticated step which uses a ``cosmetic" Da Prato-Debussche trick together with shifting of indices.   The construction of model goes via integration identities, derivative identities and symmetry.  

Afterwards, we patch the solutions in small squares together to obtain a global solution. This yields a pathwise rough Uhlenbeck compactness theorem which is neater as well as more natural. Without the additional work, or if we were to lean towwards using the black box theory, we would have a more probabilistic statement. Instead, currently, the probabilistic argument  is only used in constructing a suitable version of rough additive functions for the Yang-Mills measure. 
%




\subsection{State space for 2D Yang-Mills via Dirac operators}
%
\textit{This is joint work with Ilya Chevyrev (UoE) and Massimiliano Gubinelli (UoO).} 

\medskip 

\noindent Let $G\subset U(N)$ be a compact Lie group. Let us consider a trivial principal $G$-bundle with base manifold $\bT^2$. We can consider an associated vector bundle via a representation $G\to \operatorname{GL}(\mathbb C^N)$. It is known that connection forms $A$ can be viewed as a covariant derivative on a suitable associated vector bundle. In this project, we study properties of the space of connection forms via using covariant derivative $\diff_A$, or rather more precisely, the operator  $\mathrm{D}_A:=\diff_A\oplus\diff_A^*$ acting on the exterior algebra $\Omega=\bigoplus_{k=0}^ 2\Omega^k(\bT^2,\mathbb C^N)$. 

For a Banach space $X$ of functions we denote by $\Omega X$ the space of differential forms constitued by functions in $X$. We view the operator $\rmD_A$ as an unbounded operator on $\Omega L^2$. We define a suitable domain for $\rmD_A$ to make it a closed, self-adjoint operator with compact resolvent. We can show that the spectrum consists of discrete set eigenvalues with finite multiplicities. We can use suitable resolvent estimates and expansions to show that the orbit space $\Omega C^{\alpha-1}/\mathfrak G^\alpha$ is Polish.

Furthermore, we can establish a natural gauge invariant observables via spectral properties  in the case of $G=U(N)$ and recover gauge transformations via these observables in this case.  



\section{Future Research Directions}
In the future, I am planning to extend the results of my PhD thesis to more sophisticated settings. Broadly speaking, I am always open to explore directions in rough analysis that are not necessarily mentioned in this list. 

%The general problems I would like to work on are those in intersection of SPDEs, rough analysis, functional analysis and geometry. 

%General explanation, functional analysis, probability interaction, spdes, pdes, finding more structures, generalising ideas, global existence 

\subsection{Rough Uhlenbeck compactness on closed surfaces}
%
This project builds on the rough Uhlenbeck compactness in my PhD thesis, extending them to a generic closed smooth surface $M$ as the base manifold of the trivial principal $G$-bundle. I have been developing this in parallel, with the possibility of including preliminary findings in my thesis.
%
%This project is generalising the previous work Rough Uhlenbeck in my PhD thesis, but instead consider a generic closed smooth surface $M$ as base manifold. I have been working on this in parallel and it could possibly appear in my PhD thesis.
%

As in the classical setting, the main work involves proving the existence of the Coulomb gauge on a small Euclidean ball; in our setting, however, it suffices to consider a small Euclidean square. We need to generalise the metric of connection forms we have to use. The idea is to consider a suitably graph of the manifold and consider the supremum of the metric on each face (as the faces are homeomorphic to squares). We have to make sure that each face of the graph is small enough (in fact smaller than the injectivity radius). On each face, we use the techniques developed in the PhD thesis. We consider a further a subgraph on each face. Each subfaces is then mapped to a Euclidean square on which we have the smallness assumptions to find a Coulomb gauge as done in the rough Uhlenbeck compactness on the unit square. 

The main difficulty in applying this to Yang-Mills theory lies in the fact that, unlike the square case, the lasso field—or effectively, the curvature—does not behave as white noise. Instead, we work with a conditional measure on the initial graph of the manifold. On each face, we introduce a lasso field and derive bounds based on the conditional measure, which, while not strictly Gaussian, possesses comparable Gaussian bounds.
%
\subsection{State space for 3D Yang-Mills via Dirac operators}
%
Even though, the (non-Abelian) 3D Yang-Mills measure is not constructed yet, it is very relevant to study a potential space where the measure is supported on. In fact, to construct the measure via the stochastic quantisation, it is crucial to have a state space, e.g.\ Polish, to hope that one has nice (Markovian) properties and possibly invariant measure for which it converges to. One needs to somehow obtain a gauge group $\fG$ that acts on the possible state space $\cS$ that the dynamic takes values in. This gauge group is likely be depending on the element $s\in\cS$, which gives it likely a fibre bundle structure. There is a candidate space in \cite{CCHS3D} which is lacking a gauge group $\fG$. There is an equivalence relation, resembling classical gauge equivalence, defined by means of the Yang-Mills heat flow. This approach could poissbly be developed further by incorporating on one hand rough line integration, i.e.\ rough additive functions, from rough Uhlenbeck compactness and on the other hand the covariant derivative from \orange{REf??}. Even though, incorporating rough additive functions is something my advisor Ilya and I have tried during my PhD, we realised there is ``more structure'' that we somehow were missing. 

Using the Dirac operator approach, we hope to have an alternative approach. As we have seen in \orange{Sec?}, we were able to prove Polishness of the quotient space by merely using resolvent analysis. While such approach does not naturally carry over, I hope it is still worth exploring its boundaries. In fact, one difficulty is that the connection form in 3D is expected to be of regularity $C^{-1/2^-}$ which makes the definition of $\rmD_A$ really a singular operator in the sense that the resolvent equation is a singular (S)PDE. Luckily, this by itself will not cause issues as we know how to solve it. The issue is that the gauge transformation also require additional structure. This structure is not clear yet. On the other hand the gauge transformation on covariant derivative is $g\rmD_Ag^{-1}$ which as an operator makes sense without too much regularity on $g$. For example, on $L^2$, we can simply take $g:\bT^2\to G$ measurable. We do not exclude that for the gauge transformation one might need the theory of rough additive functions. It is likely that combining these two approaches will maximise the gain. 

\subsection{Covariant GFF and constructing conditional Higgs measure}
With Dr.\ Ajay Chandra, we are planning to study conditional Higgs measure, by first studying a covariant GFF. That is, for a given (rough) $A$, we would like to study the Gibbs-type measure
\[
\mu_A(\diff\Phi)=\frac 1 {Z_A} \exp\left( -\int_{\bT^2}|\diff_A\Phi(x)|^2+c_A|\Phi(x)|^2\diff x\right)\diff\Phi. 
\]
Of course, the way it is written, this measure is ill-defined. This measure is representing a Gaussian free field with covariance kernel given by the Green function for $(\diff_A^*\diff_A+c_A)$. There are several ideas that one could try: study the measure via the covariant Laplacian, Dirichlet form or through the Feyman-Kac formulation of the covariance kernel. 

We first try to construct this measure in the Abelian case and later extend it to non-Abelian setting. The way to rigorously construct this measure is by first taking a mollified $A^\varepsilon\to A$ and consider the measure $\mu_{A^\varepsilon}$. In the application we have in mind $A$ is essentially Gaussian free field. In this specific setting, we expect to need to take $c_{A^\varepsilon}\to-\infty$ as a renomralisation (indeed, this can be seen by writing the covariant Laplacian in coordinates). 

Studying of the covariant Laplacian requires to set up a suitable domain and studying the spectral properties. This is very similar to what I have worked on in \orange{Ref??} (but the operators considered are similar but also different). It usually common to alternatively study the Dirichlet form corresponding to the Laplacian. In our setting, it would be 
\[
\mathcal E_A(\varphi,\psi)=(\diff_A\varphi,\diff_A\psi)+c_A(\varphi,\psi).
\]
The issue that could arise is that this operator is non-degenerate as we are going to take $c_A< 0$. 

The last approach is relying on the works by \orange{REF??} and study integrals of the form $\int A(B_t)\cdot\diff B_t$ for the connection form $A$, say $\operatorname{div} A=0$ to make this integral coincide with Stratonovich. This integral is not well-defined via Ito-calculus, but one might wonder whether there is a procedure combining rough paths and stochastic sewing to make sense of this integral. This allows to define stochastic parallel transport which is needed for the formula of the Gaussian free field. The number $c_A$ should somehow appear in the formulas and that could save the day. 


%On the level of the operator $(\diff_{A^\varepsilon}^*\diff_{A^\varepsilon}+c_{A^\varepsilon})$ it somehow corresponds to moving the spectrum downwards. Indeed, once one expands the covariant Laplacian a term $A^2$ appears which, morally, is shifting the spectrum upwards, so the hope is to ``renoramlise'' it by shifting it downwards with $c_A$.  operator

%\subsection{Construction (covariant) Gaussian free field via conditioning on submanifolds}
%Or generally understanding how the Higgs field reacts to patching procedures... 


%\subsection{Global existence for non-Abelian YMH}

\subsection{Alternative proof for rough Uhlenbeck compactness in 2D}

%Natural quantities, ... Langevin dynamic for example, from Dirac operators?, also helpful for 3D if one finds a suitable way, global existence for pure 2D YM etc, we use the structure of 2D, Broux-Otto-Steele techniques, or only Otto-et al ``energy'' technique


%Alternative proof for rough Uhlenbeck compactness is not for the sake of the insane amount of possibilies to reporve the same result. It will teach us more on  singular SPDEs techniques,  rough differential geometry,and gauge invariant quantities.

The motivation for developing an alternative proof of rough Uhlenbeck compactness is not merely to replicate the result through numerous methods. Instead, this enhances our understanding of singular SPDE techniques, rough differential geometry, and gauge-invariant quantities. For example, one could try to prove using techniques relying on implicit function theorem as classically done in \orange{REf??} which is not currently available for singular SPDEs (even though there is someone working on such thing \orange{???}). In fact, in the work mentioned in \orange{REf?}, we developed many interesting techniques to arrive to the result. 

The gauge-invaraint observable that we use is unfortunately not easily applicable for the Langevin dynamic. Changing the observable to a different quantity and reporving the same result could be more natural for the Langevin dynamic. One example could be the $L^p$-norm of the curvature smoothed by the Yang-Mills heat flow with some blow-up controlled suitably in the time parameter. Another one could be spectral quantities that can be obtained from Dirac operator $\rmD_A$, for example trace of the heat kernel. 

Further benefits for tweaking the result to accommodate the 2D Langeivn dynamic is that it could initiate proving pathwise global existence of the orbits under the dynamic as well as global existence for the non-Abelian Yang-Mills-Higgs. Furthermore,  it could enhance our understanding of the 3D Yang-Mills measure. For instance, the result that we currently have in \orange{Ref?} using the axial gauge relation to lasso field which is very special property of 2 dimensions (in fact, we know even on $\bT^2$, we do not have such relation).  


\subsection{Exploring rough Calderon problems}
This is a more exploratory project. Calderon type problems, or inverse problem already arises in the study of the operator $\rmD_A=\diff_A\oplus\diff_A^*$ from \orange{Ref??}. One could try to obtain information of $A$ by studying $\rmD_A$, as well as stability results, i.e.\ some kind of continuity of the map $\rmD_A\mapsto A$. There is many results for operators of the form $\Delta+\zeta$ as well as recently for covariant Laplacian $\diff_A^*\diff_A$ and characterising up to gauge transformation \orange{REf??}. These questions are not precisely the same as the one I have mentioned, but it regardless poses the question: what happens when $\zeta$ is a 2D white noise, or $A$ is Gaussian free field on $\mathbb T^2$? Of course, there are issues: firstly the corresponding equations require singular SPDE technique, secondly boundary conditions, specifically, the Dirichlet-to-Neumann map, might require special effort to be understood. 





%\subsection{Regularity structures incorporating initial data issues}
%Also what about spatial boundaries? These could also cause similar issues, how to handle? We have a similar situation in RUC. 







%\subsection{Revisiting Unsolved Problems}
%During the first period of my PhD I have worked on several interesting problems that did not go anywhere. There are two of them which I want to revisit. 

%Lasso fields played an important role in RUC. One could ask whether lasso fields are well-defined as distribution for elements in $C^{0^-}$ (namely the same regularity as GFF). For instance, it is not clear for the YM Langevin dynamic whether there is a way of making sense of the lasso fields in a pathwise manner. The question here, whether one can find a topology for which the GFF lives in, and one can canonically define the lasso field. What we know from pathwise techniques, is that one should aim to find an expansion in terms of stochastic objects, think of modelled distributions or germs. The issue that we have previously faced was that the expansion of the lasso field has functions of the lasso field in the expansion. Alternatively, one can write an equation in terms of two dimensional rough integration where the ``noise'' depends on the solution. Of course this is a bad sign. 

%The new idea is to really use the fact that one already knows the lasso field in the smooth setting. One could hope for the smooth approximation to converge via smallness assumptions of the equantities involved. 


 


%\section{Broader Impact and Applications}


\end{document}


