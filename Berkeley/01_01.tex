\documentclass[./Research_statement.tex]{subfiles}

\begin{document}
\begin{itemize}
    \item A=Ilya Chevyrev
    \item B=Tom Klose
    \item C=Massimiliano Gubinelli
    \item D=Ajay Chandra
\end{itemize}

07/2025

\medskip 

I have worked on an extension of Uhlenbeck compactness \orange{Ref?} for two dimensional base manifolds with connection forms of distributional regularity. The regularity coincides with the regularity that can be sampled from the 2D Yang-Mills measure which as a corollay yields a gauge fixing procedure for the 2D Yang-Mills measure. The first paper in that direction is a joint work with A and B where the base manifold is the unit square $M=[0,1]^2$. 

\paragraph{Norms}
Classical Uhlenbeck compactness uses $W^{1,p}$ norm of the connection form $A$ and $L^p$ norm of the curvature $F^A=\diff A+[A\wedge A]$. The result states $\|A^g\|_{W^{1,p}}\lesssim \|F^A\|_{L^p}$ where $A^g$ corresponds to ``coordinate" transformed $A$ by so-called gauge transformation $g$. In the setting of Yang-Mills, one expects to have $F^A\in C^{-1^-}$ and as such the $L^p$ norm is not well-defined. We had to find a suitable norm. 

The suitable norm is called rough additive functions (it relates to the curvature). But we realise it is related to the axial gauge etc this goes in a different direction+Kolmogorov. 

\paragraph{Coulomb condition + Boundary + identification}
We got inspired from the classical theory and we try to find $A^g$ such that $\diff^*A^g=0$ on  a small enough square $\Lambda^\sigma=[0,\sigma]^2$. This yields a singular SPDE which we solve with theory of regularity structures. However, one challenge is that there are many degrees of freedom to write the same equation. There is also boundary conditions, which normally is simply taken to be Dirichlet or Neumann zero, which in our case were not always useful. Even though, it seems that this opens a lot of opportunities, not all equations were equally treatable. Even more so, because these equations are all singular.
For example, some equation has a highly non-local function of $A^g$, a bad boundary singularity, or another one is too singular due to derivative of ``noise", etc.  There is an identification step, which basically says that $A^g$ is genuinely a coordinate transformation of $A$. This step is highly sensitive of the equation in the interior and boundary conditions. At some point, I went through these degrees of freedom to get a feeling on which one does the job for us. 



\paragraph{Regularity structures + Picard iteration}
As previously mentioned, the equation is solved with regularity structures. There are several consideration that hinder us from applying the black box theory \orange{Ref??}. The main one is that we want to achieve a very specific result, i.e.\ a pathwise bound on the model via our norm $\vertiii{A}$. There is also a lack of boundary regularity which requires treatment. 

To obtain the model bounds, we need to work in subspaces of elements that contain symmetry. This requires to adapt the Picard iteration in such a way that those symmetries are preserved. This is a non-trivial task which requires an additional Da-Prato Debussche argument.

\paragraph{Model bounds+boundary modelled considerations}
The construct the model via the $A$.


\paragraph{Patching}

\paragraph{Generic Manifold case}


\subsection{Covariant derivative}





\end{document}