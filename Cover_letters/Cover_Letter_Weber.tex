% Author: Abdulwahab Mohamed
%
\documentclass[12pt,a4paper]{moderncv}      
\usepackage[english]{babel}

\moderncvstyle{classic}                           
\moderncvcolor{black}   


% character encoding
\usepackage[utf8]{inputenc}                     
\usepackage{ragged2e}

% adjust the page margins
\usepackage[scale=0.80]{geometry}



% personal data
\name{Abdulwahab Mohamed}{}
\email{a.mohamed@ed.ac.uk}

\begin{document}

\recipient{University of Münster}{M\"unster, Germany}
\date{\today}
\opening{Dear Comittee Members,}
\closing{Sincerely,\vspace{0em}}

%\enclosure[Enclosed]{
%\begin{enumerate}
 %   \item CV
   % \item PhD Thesis Summary
 %   \item Research Statement
    %\item Two papers in progress:\ ``Rough Uhlenbeck compactness" and ``Operator approach for 2D YM"
%\end{enumerate}
%}

\makelettertitle
\justifying
I am writing to you to apply for the position of Postdoctoral Research Associate in the Research Training Group (RTG) at the University of Münster. I am currently finishing my PhD studies supervised by Ilya Chevyrev at the University of Edinburgh expected to finish this summer. %I was previously at Eindhoven University of Technology, where I completed my BSc and MSc. I have also been to the University of Bonn for my year abroad, where I deepened my knowledge of non-linear PDEs and stochastic analysis. My MSc thesis was on the multimarginal Schrödinger problem, supervised by Alberto Chiarini (currently at the University of Padova) and Oliver Tse (TU Eindhoven).

My research interests lie in stochastic analysis, with a particular focus on problems that strongly interact with PDEs, such as SDEs and SPDEs. During my PhD, I developed expertise in frameworks such as regularity structures and paracontrolled calculus to study (singular) SPDEs. My main projects has been on problems arising in two- and three-dimensional Yang-Mills theory, which also involves differential geometry and gauge theory. Specifically, my PhD focuses on two projects: proving rough Uhlenbeck compactness for two-dimensional Yang-Mills theory using (singular) SPDE techniques, 
%(in collaboration with  Ilya Chevyrev and Tom Klose)
 and exploring an operator-based approach to Yang-Mills theory via the covariant derivative. %(in collaboration with Ilya Chevyrev and Massimiliano Gubinelli).  
%
My first project introduces several novelties, such as the deterministic construction of singular objects through a simpler object within the framework of regularity structures, which typically rely on stochastic estimates. These results, which we expect to publish in a leading journal soon, contribute significantly to the mathematical understanding of Yang-Mills theory and open doors for new techniques in singular SPDEs.
%
These projects, along with my participation in a reading group on the flow equation approach within Nicolas Perkowski’s research group at FU Berlin, have provided me with a solid foundation in stochastic analysis, rough analysis, PDEs, gauge theory, and functional analysis.



%While my work is ongoing, the first project introduces several novel techniques, such as a deterministic construction of regularity structures that typically rely on stochastic estimates. These results, expected to lead to a strong publication, not only contribute to the mathematical understanding of Yang-Mills theory but also pave the way for further exploration in related areas


The proposed projects in the attached research plan focus on singular SPDEs, a topic explicitly mentioned in the RTG's research areas. In particular, singular SPDEs are closely connected to interacting particle systems and stochastic homogenisation, two other topics within the RTG. I am also enthusiastic about collaborating across the RTG’s research areas, where I believe my expertise can contribute meaningfully.

In addition to my research interests, I have actively participated in academic events such as workshops and summer schools where I also gave talks. I have organised and led reading groups on Malliavin calculus and rough paths, aimed at connecting PhD students across different areas of probability. These efforts reflect my commitment to creating a collaborative and stimulating research environment, aligning with the RTG’s goals of building a vibrant research community. %Additionally, my teaching experience at the University of Edinburgh and Eindhoven University of Technology, where I obtained my BSc and MSc, enhanced my ability to communicate complex ideas effectively.


%My knowledge in regularity structures, quantum gauge theory, and stochastic analysis directly aligns with the research in your research group, and I believe my experience therein could meaningfully contribute to your group. Other groups at EPFL, such as the community in  stochastic analysis, random geometry, and mathematical physics, makes EPFL an ideal environment to discuss and collaborate with experts in these areas.


I have enclosed my CV, PhD thesis summary and research plan. References will be provided by Ilya Chevyrev, Alberto Chiarini and Oliver  Tse. Please do not hesitate to contact me if you require any further information. Thank you for considering my application. %I look forward to contributing to EPFL, collaborating with you, and learning from you and its research community.






\vspace{0.5cm}

\makeletterclosing

\end{document}
