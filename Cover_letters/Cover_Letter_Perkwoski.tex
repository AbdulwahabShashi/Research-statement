% Author: Abdulwahab Mohamed
%
\documentclass[11pt,a4paper]{moderncv}      
\usepackage[english]{babel}

\moderncvstyle{classic}                           
\moderncvcolor{black}                            

% character encoding
\usepackage[utf8]{inputenc}                     

% adjust the page margins
\usepackage[scale=0.80]{geometry}

% personal data
\name{Abdulwahab Mohamed}{}
\email{a.mohamed@ed.ac.uk}

\begin{document}

\recipient{Max Planck Institute for Mathematics in the Sciences}{Leipzig, Germany}
\date{\today}
\opening{Dear Prof.\ Dr.\  Nicolas Perkowski,}
\closing{Kind regards,\vspace{-2em}}
\enclosure[Enclosed]{{\footnotesize 
\begin{enumerate}
    \item CV
    \item PhD Thesis Summary
    \item Research Statement
    \item Two papers in progress:\ ``Rough Uhlenbeck compactness" and ``Operator approach for 2D YM"
\end{enumerate}
}
}

\makelettertitle

I am writing to you to apply for the postdoc position at the Max Planck Institute for Mathematics in the Sciences. I am currently finishing my PhD studies supervised by Dr.\ Ilya Chevyrev at the University of Edinburgh. I was previously at Eindhoven University of Technology, where I completed my BSc and MSc. I have also been to the University of Bonn for my year abroad, where I deepened my knowledge in non-linear PDEs and stochastic analysis. My MSc thesis was on the multimarginal Schrödinger problem, supervised by Dr.\ Alberto Chiarini (currently at the University of Padova) and Dr.\ Oliver Tse (TU Eindhoven).

My research interests lie in stochastic analysis, with a particular focus on problems that strongly interact with PDEs, such as SDEs and SPDEs. During my PhD, I developed expertise in frameworks such as regularity structures and paracontrolled calculus to study (singular) SPDEs. My primary work has been on problems arising in two- and three-dimensional Yang-Mills theory, which also involves differential geometry and gauge theory. Specifically, my PhD focuses on two key projects: proving rough Uhlenbeck compactness for two-dimensional Yang-Mills theory using (singular) SPDE techniques, in collaboration with Dr.\ Ilya Chevyrev and Dr.\ Tom Klose, and exploring an operator-based approach to Yang-Mills theory via the covariant derivative, in collaboration with Dr.\ Ilya Chevyrev and Prof. Dr.\ Massimiliano Gubinelli. These projects have provided me with a solid foundation in stochastic analysis, rough analysis, PDEs, gauge theory, and functional analysis.

Additionally, I initiated side projects with other PhD students on regularisation by noise for infinite-dimensional equations, including exploring extensions of stochastic sewing by Matsuda and yourself. Although these projects did not lead to publications, they deepened my appreciation of the challenges of extending stochastic techniques to infinite-dimensional settings.

This postdoc position aligns naturally with my research interests and offers a great opportunity to explore new directions that build on my expertise in SPDE techniques, rough analysis, and functional analysis. While I have not directly worked on the precise area you specialise in, the methods and frameworks I have developed during my PhD allow me to transition smoothly into other problems in the broader area of stochastic analysis. The Max Planck Institute’s research community in (singular) SPDEs and quantum field theory offers a great environment to deepen my understanding and collaborate with experts in these areas.

Thank you for considering my application. I look forward to contributing to the Max Planck Institute for Mathematics in the Sciences, collaborating with you, and learning from you and its research community.


\vspace{0.5cm}

\makeletterclosing

\end{document}
