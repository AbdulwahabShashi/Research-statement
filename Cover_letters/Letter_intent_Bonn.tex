% Author: Abdulwahab Mohamed
%
\documentclass[12pt,a4paper]{moderncv}      
\usepackage[english]{babel}

\moderncvstyle{classic}                           
\moderncvcolor{black}   


% character encoding
\usepackage[utf8]{inputenc}                     
\usepackage{ragged2e}

% adjust the page margins
\usepackage[scale=0.80]{geometry}



% personal data
\name{Abdulwahab Mohamed}{}
\email{a.mohamed@ed.ac.uk}

\begin{document}

\recipient{Hausdorff Center for Mathematics}{Bonn, Germany}
\date{\today}
\opening{Dear organisers,}
\closing{Kind regards,\vspace{0em}}

%\enclosure[Enclosed]{
%\begin{enumerate}
 %   \item CV
   % \item PhD Thesis Summary
 %   \item Research Statement
    %\item Two papers in progress:\ ``Rough Uhlenbeck compactness" and ``Operator approach for 2D YM"
%\end{enumerate}
%}

\makelettertitle
\justifying

I am excited to apply for the Summer School ``Probabilistic Methods in Quantum Field Theory", which aligns closely with my research focus on stochastic analysis and quantum gauge theory. I am currently in the last year of my PhD  under Ilya Chevyrev. 
%
%I was previously at Eindhoven University of Technology, where I completed my BSc and MSc. I have also been to the University of Bonn for my year abroad, where I deepened my knowledge of non-linear PDEs and stochastic analysis. My MSc thesis was on the multimarginal Schrödinger problem, supervised by Alberto Chiarini (currently at the University of Padova) and Oliver Tse (TU Eindhoven).
%
My research interests lie in stochastic analysis with particular focus on problems  that strongly interact with PDEs and mathematical physiscs. 
%
%During my PhD, I developed expertise in frameworks such as regularity structures and paracontrolled calculus to study (singular) SPDEs. 
%
My PhD thesis centers on quantum gauge theory, particularly within two- and three-dimensional Yang-Mills theory. Specifically, my PhD focuses on two projects:\ rough Uhlenbeck compactness for two-dimensional Yang-Mills theory using (singular) SPDE techniques, in collaboration with  Ilya Chevyrev and Tom Klose, and exploring an operator-based approach to Yang-Mills theory via the covariant derivative, in collaboration with Ilya Chevyrev and Massimiliano Gubinelli. 



I would like to attend this summer school to deepen my understanding of probabilistic approaches to quantum field theory and to explore topics outside my current expertise. While my research has mainly focused on continuum Yang-Mills theory, I am particularly looking forward to learning about the topics in the description and benefiting from the expertise of the lecturers. Specifically, I am interested in learning more about lattice Yang-Mills-Higgs theory, CFT and SLE, and their connections. The latter would allow me to obtain first steps towards a new expertise that I would like to develop for my academic career.  
%I would like to attend this summer school as I will get the opportunity to learn new aspects and approaches within probabilistic quantum field theory. I have noticed that some of the lecturers are experts on lattice Yang-Mills-Higgs theories, which I would like to learn about as I have mainly  worked in the continuum myself. Furthermore, some of the  lecturers are experts in SLE and CFT which are topics that I have not personally worked on but I am excited  to learn about as first steps to broaden my expertise.

Besides the lectures, I am looking forward to discuss with the participants as well as the researchers who are staying for the trimester program. %For example, I am particularly excited about the possibility of meeting Ajay Chandra to discuss about our project on conditional Higgs measure during the stay. 
In general, I believe these discussions could lead to enhanced network and collaborations within quantum gauge theory or related fields. 

I would be grateful to receive funding for accommodation as this support would enable me to participate in this summer school. Unfortunately, my university's travel funding would only allow me to cover transportation costs. 

%In these projects, I have proved several new and, in some cases, surprising results. For example, in the first project, I constructed the model and several auxiliary input data in the regularity structures framework using a much simpler object deterministically. Typically, however, such constructions rely on stochastic estimates. In the second project, one of the results I have proved is the Polishness of the gauge orbit space for the 2D Yang-Mills measure using only Hölder-Besov norms, in contrast to stricter norms used in the paper by my supervisor, yourself and collaborators. This proof uses properties of the resolvent of a Dirac operator associated with the covariant derivative. These projects, along with my participation in a reading group on the flow equation approach within Nicolas Perkowski’s research group at FU Berlin, have provided me with a solid foundation in stochastic analysis, rough analysis, PDEs, gauge theory, and functional analysis.

%My long-term research vision is to make further advancements in quantum gauge theory and generally rough analysis. For example, developing an operator approach to Yang-Mills theory which would provide an alterantive viewpoint. 


%In these projects, I have independently proved several new, and some even surprising, results. To mention a few, in the first project, I managed to construct the model and several auxiliary input data coming from boundary in the regularity structures framework using a much simpler---geometrically more natural---object, while normally one would use stochastic estimates for those. In the second project, I have proved Polisheness of gauge orbit space for the 2D Yang-Mills measure using merely Holder-Besov norms, while previous works by among other, my supervisor  and yourself, required stricter norms. The proof relies on using properties of the resolvent. These projects, along with my participation in a reading group on the flow equation approach within Nicolas Perkowski’s research group at FU Berlin, have provided me with a solid foundation in stochastic analysis, rough analysis, PDEs, gauge theory, and functional analysis. 


%I have attended several conferences, workshops, and summer schools, where I had the opportunity to present my ongoing work. Furthermore,  I have gained teaching experience through leading tutorial sessions at the University of Edinburgh and Eindhoven University of Technology on courses such
%as calculus, probability, and advanced postgraduate courses. I enjoy teaching and sharing knowledge, and I would be happy to contribute to teaching responsibilities if needed. Finally, I have organised two reading groups, one on Malliavin calculus, and the other on rough paths with signatures in machine learning. The aim was, besides learning, to build a group of PhD students with shared interests in probability. Even though I was already familiar with rough paths, I initiated the reading group to introduce this strong yet lesser-known theory to colleagues across different areas of probability (and analysis). My aim was to show its usefulness and create opportunities for collaboration between different areas of probability. These experiences show my commitment to taking part in departmental activities and trying to bring people together to work on shared interests, aligning with the responsibilities outlined for this position. 

%Additionally, I initiated side projects with other PhD students on regularisation by noise for infinite-dimensional equations, including exploring extensions of stochastic sewing by Matsuda and Perkwoski as well as stochastic reconstruction theorem by Hannes Kern. While these projects did not lead to publications, they provided valuable opportunities to engage with advanced stochastic sewing techniques.


%My knowledge in regularity structures, quantum gauge theory, and stochastic analysis directly aligns with the research in your research group, and I believe my experience therein could meaningfully contribute to your group. Other groups at EPFL, such as the community in  stochastic analysis, random geometry, and mathematical physics, makes EPFL an ideal environment to discuss and collaborate with experts in these areas.


Finally, the  reference will be provided by Ilya Chevyrev whose e-mail is ichevyrev@gmail.com.   Please do not hesitate to contact me if you require any further information. Thank you for considering my application. %I look forward to contributing to EPFL, collaborating with you, and learning from you and its research community.






\vspace{0.5cm}

\makeletterclosing

\end{document}
