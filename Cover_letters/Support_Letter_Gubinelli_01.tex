\documentclass[12pt]{article}
\usepackage[utf8]{inputenc}
\usepackage[small]{titlesec}
\usepackage{fullpage}
\usepackage{amsmath,amsthm}
\usepackage{amssymb}
\usepackage{mathrsfs}
\usepackage{bbm}
\usepackage{comment}
\usepackage{subfiles}
\usepackage{enumitem}
\usepackage{graphicx}
\usepackage{color}
\usepackage{xcolor}
\usepackage{tcolorbox}
\usepackage{float}
\usepackage{xr-hyper}
\usepackage[hidelinks]{hyperref}
\usepackage{cleveref}
\numberwithin{equation}{section}
\usepackage{float}


%\usepackage{bibtex}
%\usepackage[
 %   backend=biber,
 %   style=alphabetic,
 %   maxnames=50,
 %   firstinits=false,
 % ]{biblatex}

%\addbibresource{references.bib}

\newtheorem{theorem}{Theorem}[section]
\newtheorem*{theorem*}{Theorem}
\newtheorem{corollary}[theorem]{Corollary}
\newtheorem{lemma}[theorem]{Lemma}
\newtheorem{proposition}[theorem]{Proposition}
\theoremstyle{definition}
\newtheorem{definition}[theorem]{Definition}
\newtheorem{example}[theorem]{Example}
\newtheorem{assumption}[theorem]{Assumption}

\theoremstyle{remark}
\newtheorem{remark}[theorem]{Remark}
\definecolor{brown}{rgb}{0.5, 0.21, 0.15}

\definecolor{darkgreen}{rgb}{0.05, 0.7, 0.06}
\newcommand{\brown}[1]{\textcolor{brown}{#1}}
\newcommand{\PM}{\mathbb{P}}
\newcommand{\E}{\mathbb{E}} % \E abbreviation for expectation
\newcommand{\Log}{\operatorname{Log}}
\newcommand{\hol}{\mathrm{hol}}
\newcommand{\Ad}{\mathrm{Ad}}
\newcommand{\id}{\mathrm{id}}
\newcommand{\dif}{\,\mathrm{d}}
\newcommand{\diff}{\mathrm{d}}
\newcommand{\R}{\mathbb R}
\newcommand{\Law}{\mathrm{Law}}
\newcommand{\1}{\mathbf 1}
\newcommand{\<}{\langle}  
\renewcommand{\>}{\rangle}
\newcommand{\sign}{\operatorname{sgn}}

\newcommand{\hor}{\text{-}\mathrm{hor}}
\newcommand{\hgr}{\text{-}\mathrm{hgr}}
\newcommand{\ax}{\text{-}\mathrm{ax}}
\newcommand{\gr}{\text{-}\mathrm{gr}}



\definecolor{blueblue}{rgb}{0.3, 0.3, 0.95}
\newcommand{\red}[1]{\textcolor{red}{#1}}
\newcommand{\blue}[1]{\textcolor{blueblue}{#1}}
\newcommand{\green}[1]{\textcolor{darkgreen}{#1}}
\newcommand{\orange}[1]{\textcolor{orange}{#1}}

\newcommand{\bfA}{\mathbf A}
\newcommand{\bA}{\mathbb A}
\newcommand{\bfOmega}{\boldsymbol{\Omega}}

\newcommand{\cA}{\mathcal A}
\newcommand{\cB}{\mathcal B}
\newcommand{\cD}{\mathcal D}
\newcommand{\cF}{\mathcal F}
\newcommand{\cG}{\mathcal G}
\newcommand{\cH}{\mathcal H}
\newcommand{\cI}{\mathcal I}
\newcommand{\cJ}{\mathcal J}
\newcommand{\cK}{\mathcal K}
\newcommand{\cL}{\mathcal L}
\newcommand{\cP}{\mathcal P}
\newcommand{\cR}{\mathcal R}
\newcommand{\cS}{\mathcal S}
\newcommand{\cT}{\mathcal T}
\newcommand{\cU}{\mathcal U}
\newcommand{\cX}{\mathcal X}
\newcommand{\cZ}{\mathcal Z}

\newcommand{\bHa}{\mathbb{H}_{\mathsf{a}}}
\newcommand{\bHc}{\mathbb{H}_{\mathsf{c}}}

\newcommand{\rmD}{\mathrm{D}}

\newcommand{\fG}{\mathfrak{G}}
\newcommand{\fg}{\mathfrak g}

\newcommand{\bC}{\mathbb C}
\newcommand{\T}{\mathbb T}
\newcommand{\bH}{\mathbb H}
\newcommand{\bN}{\mathbb N}
\newcommand{\bP}{\mathbb P}
\newcommand{\bQ}{\mathbb Q}
\newcommand{\bS}{\mathbb S}
\newcommand{\bT}{\mathbb T}
\newcommand{\bZ}{\mathbb Z}
\newcommand*\recvert[1]{\left[\!\left]#1\right[\!\right]}

\newcommand{\vertiii}[1]{{\left\vert\kern-0.4ex\left\vert\kern-0.4ex\left\vert #1 
    \right\vert\kern-0.4ex\right\vert\kern-0.4ex\right\vert}}
\newcommand{\triple}[1]{\vertiii{#1}}



    
%\title{Research Statement}
%\author{Abdulwahab Mohamed}
%\author{***}


\begin{document}

%\maketitle
\vspace{-10pt}
\begin{center}
    {\Large \textbf{Supporting Statement for Postdoctoral Position}} \\ \vspace{1pt}
    \textbf{Vacancy reference:} 176890\\
      Abdulwahab Mohamed \vspace{-0.5cm}
\end{center}

\section{Introduction}
My name is Abdulwahab Mohamed, and I am currently completing my PhD in Mathematics at the University of Edinburgh. My research interests lie in stochastic analysis, particularly (singular) SPDEs and quantum gauge theory. My PhD research aligns closely with the research focus of Prof.\  Massimiliano Gubinelli. This supporting statement outlines how my background and experience fulfill the selection criteria for the postdoctoral position in Prof.\ Gubinelli's group.

\section{Essential Selection Criteria}
\subsection{Have, or be close to completing, a PhD in Mathematics or Physics}
I am nearly completing my PhD in Mathematics at the University of Edinburgh, with my viva expected in July 2025. My PhD thesis focuses on problems in two-dimensional Yang-Mills theory, in particular rough Uhlenbeck compactness, as well as exploring an operator approach to Yang-Mills theory.

\subsection{Possess sufficient specialist knowledge in the discipline to work within established research programmes}
My research has required understanding and learning a recent field within stochastic analysis, namely rough analysis, particularly frameworks such as regularity structures and paracontrolled calculus to study singular SPDEs. Furthermore, as my problems are related to Yang-Mills theory, I have learned differential geometry and gauge theory. Moreover, the operator approach gave me an opportunity to revise elements from functional analysis, e.g., unbounded operators and spectral theory. 

Additionally, I participated in a reading group on the flow equation approach organised by Prof. Nicolas Perkowski at FU Berlin, where I presented a section. This experience provided insight into recent methodologies in constructive QFT, such as those discussed in "Parabolic stochastic quantisation of the fractional $\Phi^4_3$ model in the full subcritical regime" by Prof. Gubinelli et al. This complements my MSc thesis which focused on the multimarginal Schrödinger problem. That is a stochastic control problem which is relevant to some approaches in probabilistic QFT that Prof. Gubinelli has pioneered, e.g. in the paper "A variational method for $\Phi^4_3$". This diverse background equips me to contribute effectively to ongoing research in Prof. Gubinelli's group.

\subsection{Have the ability to manage own academic research and associated activities}
Throughout my PhD, I have independently managed many aspects of my research, from setting up problems to developing rigorous mathematical proofs. I have initiated a programme that significantly strengthened the main result in my main project on rough Uhlenbeck compactness. In there, there is an auxiliary singular SPDE which we solve using regularity structures and as such requires constructing singular objects. While normally one uses stochastic estimates to construct these singular objects, I realised that it should be possible to bypass stochastic estimates. This belief is strengthened by my supervisor's work  "Yang-Mills measure on the two-dimensional torus as a random distribution," where a related result was achieved under less restrictive assumptions. With a lot of effort, I have managed to construct the singular objects through a much  simpler and natural object called rough additive functions. This construction is deterministic and is continuous. Achieving this result, which was unexpected even to my supervisor, highlights my ability to formulate and tackle challenging problems independently.  



Moreover, my independence has extended to initiating collaborations with researchers and contributing to the activities of research groups. For example, I initiated side projects with other PhD students on regularisation by noise for infinite-dimensional equations, including exploring extensions of the stochastic sewing methods. Although these projects did not lead to publications, they deepened my appreciation of the challenges involved in extending stochastic sewing techniques to infinite-dimensional settings. Additionally, I organised two reading groups—one on Malliavin calculus and the other on rough paths with applications to machine learning. These reading groups aimed to build a community of PhD students with shared interests in probability and analysis. By focusing on rough paths and machine learning, we sought to explore their potential applications to pathwise stochastic control problems, with the hope of fostering collaborations with researchers working in stochastic control. 

\subsection{Previous experience of contributing to publications/presentations, with a strong publication track record for career stage}
I have presented my work at young researchers' meetings and workshops, including several invited presentations, as well as departmental seminars. These opportunities allowed me to receive valuable feedback and refine the clarity and depth of my presentations. Although I have not yet published any papers, the novelties and techniques developed in my research are expected to yield a strong publication in a reputable journal.

\subsection{Possess the ability to contribute ideas for new research projects}
My main project, as mentioned before, requires solving a singular SPDE. This SPDE is auxiliary and there is a huge flexibility in setting up the equation. I  derived a suitable equation that allows to prove the relevant properties easily.  To strengthen the main result, I introduced a 'cosmetic' Da-Prato Debussche trick  with a non-standard Picard iteration and  a systematic framework for handling the singular objects. Additionally, I have introduced auxiliary objects to handle boundary regularity issues. This is partially due to the fact that I wanted to achieve the result mentioned in the point above. As such I had to define those objects deterministically as opposed to defining stochastically usually. This is an idea whose effecitiveness I want to explore in different equations with boundary conditions as it would have push rough analysis ideas even further.  

In the operator approach project, I quickly proved new results, including the Polishness of the gauge orbit space using resolvent estimates and a characterisation of gauge transformations via gauge-invariant observables. These contributions illustrate my ability to obtain ideas for new research projects.  

\subsection{Excellent communication skills, including the ability to write for publication, present research proposals and results, and represent the research group at meetings}
I have honed my communication skills through teaching, giving presentations, and participating in academic discussions. I have experience leading tutorials at the University of Edinburgh and Eindhoven University of Technology, covering courses like calculus, probability, and advanced postgraduate topics. At Eindhoven, I contributed to developing learning materials by writing scripts for educational videos to make postgraduate probability theory more accessible. Furthermore, I wrote the initial draft of the paper discussing rough Uhlenbeck compactness independently, showcasing my ability to organise and communicate complex research.  These experiences have helped me explain complex ideas clearly and connect with a range of audiences. Additionally, I have organised and participated in seminars, fostering collaboration and sharing knowledge. My experiences highlight my enthusiasm for teaching, mentorship, and contributing actively to the Stochastic Analysis research group.

\section{Desirable selection criteria}
\subsection{Experience of independently managing a discrete area of a research project}


\subsection{Experience in one or more of the following areas: constructive Euclidean quantum field
theory, geometric partial differential equations, renormalization group methods,
conformal field theory, singular stochastic partial differential equations}


\subsection{Experience of actively collaborating in the development of research articles for
publication}
\section{Conclusion}



\end{document}


