% Author: Abdulwahab Mohamed
%
\documentclass[12pt,a4paper]{moderncv}      
\usepackage[english]{babel}

\moderncvstyle{classic}                           
\moderncvcolor{black}   


% character encoding
\usepackage[utf8]{inputenc}                     
\usepackage{ragged2e}

% adjust the page margins
\usepackage[scale=0.80]{geometry}



% personal data
\name{Abdulwahab Mohamed}{}
\email{a.mohamed@ed.ac.uk}

\begin{document}

\recipient{École Polytechnique Fédérale de Lausanne}{Lausanne, Switzerland}
\date{\today}
\opening{Dear Prof.\ Dr.\  Martin Hairer,}
\closing{Kind regards,\vspace{0em}}

%\enclosure[Enclosed]{
%\begin{enumerate}
 %   \item CV
   % \item PhD Thesis Summary
 %   \item Research Statement
    %\item Two papers in progress:\ ``Rough Uhlenbeck compactness" and ``Operator approach for 2D YM"
%\end{enumerate}
%}

\makelettertitle
\justifying
I am writing to you to apply for the postdoc position at the École Polytechnique Fédérale de Lausanne (EPFL). I am currently finishing my PhD studies supervised by Ilya Chevyrev at the University of Edinburgh with expected viva date in summer 2025. I was previously at Eindhoven University of Technology, where I completed my BSc and MSc. I have also been to the University of Bonn for my year abroad, where I deepened my knowledge of non-linear PDEs and stochastic analysis. My MSc thesis was on the multimarginal Schrödinger problem, supervised by Alberto Chiarini (currently at the University of Padova) and Oliver Tse (TU Eindhoven).

My research interests lie in stochastic analysis, with a particular focus on problems that strongly interact with PDEs, such as SDEs and SPDEs. During my PhD, I developed expertise in frameworks such as regularity structures and paracontrolled calculus to study (singular) SPDEs. My main projects has been on problems arising in two- and three-dimensional Yang-Mills theory, which also involves differential geometry and gauge theory. Specifically, my PhD focuses on two projects: proving rough Uhlenbeck compactness for two-dimensional Yang-Mills theory using (singular) SPDE techniques, in collaboration with  Ilya Chevyrev and Tom Klose, and exploring an operator-based approach to Yang-Mills theory via the covariant derivative, in collaboration with Ilya Chevyrev and Massimiliano Gubinelli. 


In these projects, I have proved several new and, in some cases, surprising results. For example, in the first project, I constructed the model and several auxiliary input data in the regularity structures framework using a much simpler object deterministically. Typically, however, such constructions rely on stochastic estimates. In the second project, one of the results I have proved is the Polishness of the gauge orbit space for the 2D Yang-Mills measure using only Hölder-Besov norms, in contrast to stricter norms used in the paper by my supervisor, yourself and collaborators. This proof uses properties of the resolvent of a Dirac operator associated with the covariant derivative. These projects, along with my participation in a reading group on the flow equation approach within Nicolas Perkowski’s research group at FU Berlin, have provided me with a solid foundation in stochastic analysis, rough analysis, PDEs, gauge theory, and functional analysis.

%My long-term research vision is to make further advancements in quantum gauge theory and generally rough analysis. For example, developing an operator approach to Yang-Mills theory which would provide an alterantive viewpoint. 


%In these projects, I have independently proved several new, and some even surprising, results. To mention a few, in the first project, I managed to construct the model and several auxiliary input data coming from boundary in the regularity structures framework using a much simpler---geometrically more natural---object, while normally one would use stochastic estimates for those. In the second project, I have proved Polisheness of gauge orbit space for the 2D Yang-Mills measure using merely Holder-Besov norms, while previous works by among other, my supervisor  and yourself, required stricter norms. The proof relies on using properties of the resolvent. These projects, along with my participation in a reading group on the flow equation approach within Nicolas Perkowski’s research group at FU Berlin, have provided me with a solid foundation in stochastic analysis, rough analysis, PDEs, gauge theory, and functional analysis. 


I have attended several conferences, workshops, and summer schools, where I had the opportunity to present my ongoing work. Furthermore,  I have gained teaching experience through leading tutorial sessions at the University of Edinburgh and Eindhoven University of Technology on courses such
as calculus, probability, and advanced postgraduate courses. I enjoy teaching and sharing knowledge, and I would be happy to contribute to teaching responsibilities if needed. Finally, I have organised two reading groups, one on Malliavin calculus, and the other on rough paths with signatures in machine learning. The aim was, besides learning, to build a group of PhD students with shared interests in probability. Even though I was already familiar with rough paths, I initiated the reading group to introduce this strong yet lesser-known theory to colleagues across different areas of probability (and analysis). My aim was to show its usefulness and create opportunities for collaboration between different areas of probability. These experiences show my commitment to taking part in departmental activities and trying to bring people together to work on shared interests, aligning with the responsibilities outlined for this position. 

%Additionally, I initiated side projects with other PhD students on regularisation by noise for infinite-dimensional equations, including exploring extensions of stochastic sewing by Matsuda and Perkwoski as well as stochastic reconstruction theorem by Hannes Kern. While these projects did not lead to publications, they provided valuable opportunities to engage with advanced stochastic sewing techniques.


My knowledge in regularity structures, quantum gauge theory, and stochastic analysis directly aligns with the research in your research group, and I believe my experience therein could meaningfully contribute to your group. Other groups at EPFL, such as the community in  stochastic analysis, random geometry, and mathematical physics, makes EPFL an ideal environment to discuss and collaborate with experts in these areas.


I have enclosed my CV and research statement in a single file, and references will be provided by Ilya Chevyrev, Massimiliano Gubinelli and Alberto Chiarini. Please do not hesitate to contact me if you require any further information. Thank you for considering my application. %I look forward to contributing to EPFL, collaborating with you, and learning from you and its research community.






\vspace{0.5cm}

\makeletterclosing

\end{document}
