% Author: Abdulwahab Mohamed
%
\documentclass[11pt,a4paper]{moderncv}      
\usepackage[english]{babel}

\moderncvstyle{classic}                           
\moderncvcolor{black}   


% character encoding
\usepackage[utf8]{inputenc}                     
\usepackage{ragged2e}

% adjust the page margins
\usepackage[scale=0.80]{geometry}


\name{Abdulwahab Mohamed}{}
\email{a.mohamed@ed.ac.uk}

\begin{document}

\recipient{Technische Universität Berlin}{Berlin, Germany}
\date{\today}
\opening{Dear Prof.\ Dr.\ Benjamin Gess,}
\closing{Kind regards,\vspace{-2em}}
%\enclosure[Enclosed]{{\footnotesize 
%\begin{enumerate}
 %   \item CV
 %   \item PhD Thesis Summary
 %   \item Research Statement
 %   \item Two papers in progress:\ ``Rough Uhlenbeck compactness" and ``Operator approach for 2D YM"
%\end{enumerate}
%}
%}

\makelettertitle
\justifying
I am writing to you to apply for the postdoc position (Reference II-678/24) at the Technische Universität Berlin. I will soon finish my PhD studies supervised by Dr.\ Ilya Chevyrev at the University of Edinburgh with expected viva date in summer 2025. I am currently visiting him at TU Berlin. I was previously at Eindhoven University of Technology, where I completed my BSc and MSc. I have also been to the University of Bonn for my year abroad, where I learned more about non-linear PDEs and stochastic analysis. My MSc thesis was on the multimarginal Schrödinger problem, supervised by Dr.\ Alberto Chiarini (currently at the University of Padova) and Dr.\ Oliver Tse (TU Eindhoven).

My research interests lie in stochastic analysis, with a particular focus on problems that strongly interact with PDEs, such as SDEs and SPDEs. During my PhD, I developed expertise in frameworks such as regularity structures and paracontrolled calculus to study (singular) SPDEs. My main work has been on problems arising in two- and three-dimensional Yang-Mills theory, which also involves differential geometry and gauge theory. Specifically, my PhD focuses on two projects: proving rough Uhlenbeck compactness for two-dimensional Yang-Mills theory using (singular) SPDE techniques, in collaboration with Dr.\ Ilya Chevyrev and Dr.\ Tom Klose, and exploring an operator-based approach to Yang-Mills theory via the covariant derivative, in collaboration with Dr.\ Ilya Chevyrev and Prof. Dr.\ Massimiliano Gubinelli.  Central to these projects are questions of well-posedness, existence, and regularity of solutions to (singular) SPDEs. Through this work, I have built a foundation in stochastic analysis, rough analysis, PDEs, gauge theory, and functional analysis.

%In these projects well-posedness, existence and regularity of solutions to (singular) SPDEs is central questions. Hence, these projects have provided me with a solid foundation in stochastic analysis, rough analysis, PDEs, gauge theory, and functional analysis.


This postdoc position aligns naturally with my research interests and offers a great opportunity to explore new directions that build on my expertise in SPDEs and rough analysis. While I have not directly worked on stochastic conservation laws, stochastic fluid dynamics and non-equilibrium large deviations and interacting particle systems, my experience with (singular) SPDEs and non-linear PDEs in different contexts provides me with a foundation to approach these fields.  Moreover, the research community in rough analysis and SPDEs at TU Berlin and the other universities offers great opportunities for collaboration and academic growth.
%
%My current visit to TU Berlin, working with Dr.\ Ilya Chevyrev, has further broadened my perspective and exposed me to new ideas in stochastic analysis. 
%
%I am particularly interested by the broader umbrella of topics related to stochastic dynamics, including works on non-equilibrium large deviations and interacting particle systems, such as those you have explored with Benjamin Fehrman. These topics offer a natural step for me, combining familiar mathematical techniques with opportunities to delve into new, yet connected, areas.


In addition to research activities, I am keen to undertake teaching duties and supervise undergraduate and postgraduate students. While I have tutoring experience, I have not yet had the opportunity to supervise students. I would value the chance to learn how to do the latter as part of this position. Finally, as proficiency in German is mentioned, I would like to note that while I am not yet fluent, learning German is a personal and professional goal I am actively working on.  My current visit to Germany has also been a great opportunity to practice and improve my skills.




%Additionally, I noticed that proficiency in German is mentioned in the job posting. Even though I am not yet proficient, I have been actively practising and improving my German, with the intention of continuing to develop my skills further. Beyond being a practical asset, I see learning German as a personal and professional goal, enriching my experience of working in Germany.



%This postdoc position aligns naturally with my research interests and offers a great opportunity to explore new directions that build on my expertise in SPDE techniques, rough analysis, and functional analysis. While I have not directly worked on the precise area you specialise in, the methods and frameworks I have developed during my PhD allow me to transition smoothly into other problems in the broader area of stochastic analysis. The Technische Universität Berlin’s research community in rough analysis and (singular) SPDEs offers a great environment to deepen my understanding and collaborate with experts in these areas.

I have enclosed my CV and research statement in a single file, and references will be provided by Ilya Chevyrev and Alberto Chiarini. Please do not hesitate to contact me if you require any further information. Thank you for considering my application.



%Thank you for considering my application. I look forward to contributing to the Technische Universität Berlin, collaborating with you, and learning from you and its research community.



\vspace{0.5cm}

\makeletterclosing

\end{document}

