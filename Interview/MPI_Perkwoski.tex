\documentclass[12pt]{article}
\usepackage[utf8]{inputenc}
\usepackage[small]{titlesec}
\usepackage{fullpage}
\usepackage{amsmath,amsthm}
\usepackage{amssymb}
\usepackage{comment}
\usepackage{subfiles}
\usepackage{enumitem}
\usepackage{graphicx}
\usepackage{color}
\usepackage{xcolor}
\usepackage{float}
\usepackage{xr-hyper}
\usepackage[hidelinks]{hyperref}
\usepackage{cleveref}
\numberwithin{equation}{section}
\usepackage{float}


\definecolor{brown}{rgb}{0.5, 0.21, 0.15}

\definecolor{darkgreen}{rgb}{0.05, 0.7, 0.06}
\newcommand{\brown}[1]{\textcolor{brown}{#1}}


\definecolor{blueblue}{rgb}{0.3, 0.3, 0.95}
\newcommand{\red}[1]{\textcolor{red}{#1}}
\newcommand{\blue}[1]{\textcolor{blueblue}{#1}}
\newcommand{\green}[1]{\textcolor{darkgreen}{#1}}
\newcommand{\orange}[1]{\textcolor{orange}{#1}}



    
\title{Interview Questions}
\author{Abdulwahab Mohamed}
%\author{***}


\begin{document}


%\maketitle
\begin{center}
    {\Large \textbf{Interview Questions}} \\ \vspace{1pt}
    for the position with Perkowski at MPI Leipzig \\\vspace{0.5cm}
\end{center}

\section{Research-Focused Questions}
\begin{enumerate}
    \item Can you elaborate on your research background and how it aligns with our group's focus?
    
    \textit{Approach: Provide a concise overview of your research, emphasizing areas that intersect with the group's interests. Highlight specific projects or publications that demonstrate this alignment.}

    \brown{\textbf{Answer:} The group focus on singular SPDEs (Nicolas Perkwoski, Felix Otto), mathematical Physics (Nicolay Barashkov, Stefan Hollands), SPDEs (Benjamin Gess), firstly my project being in Yang-Mills theory as well as our major tool being a singular SPDE to prove a gauge fixing result}

    \item How do you envision contributing to our current projects, and what unique perspectives can you bring?
    
    \textit{Approach: Identify ongoing projects within the group and articulate how your expertise can enhance these endeavors. Propose potential research directions or methodologies that could be beneficial.}

    \brown{\textbf{Answer:} what I could bring in the group is my enthusiams first of all, but I could also bring contributions in the field of quantum gauge theory or related fields, for examples, tackle geometric problems, and I believe there are among other works by Nicolas on random unbounded operator like Anderson model, similar operators also occur in gauge theory, Dirac operator, oneof the projects we working with Max, along the same veins there are possible applications of regularisation by noise techniques that I believe could be used to understand Higgs Mechanism in Yang-Mills-Higgs theory. Moreover, with the works by Prof Felix Otto,     for example that there are interesting examples where the newly approached could benefit from. I also believe there might be more geometric problems where the concept of rough additive functions could be beneficial. }
    

     \item Tell us about your previous research.
    
    \textit{Approach: Summarize your prior research succinctly, emphasizing key findings and methodologies.}

    \brown{\textbf{Answer:} Some works with Higgs fields, additive functions, construction of lasso field, difficult and related to 2D rough integration problems, rough additive functions application to state space for 3D YM, RUC, Operator approach}
    
    \item Tell us about the work you would plan to do during this postdoc.
    
    \textit{Approach: Outline your planned research directions, linking them to the group's focus and goals.}

    \brown{\textbf{Answer:} I would like to discuss with Nicolas on the projects he has in mind as well as you and maybe other professors in the group. There are some problems I have outlined in my research statement, namely generalising RUC for arbitrary manifolds, continue developing the operator approach, this is part of big program that will not finish soon, construction or rather contribute to understanding how to construct the Higgs field. Another problem that I like to look at is singular SPDEs on bounded domains with rough boundary data, this might help us understand Markov property of $\Phi^4_3$, one of the things it could do. It would incorporate or have similar stuff iwth boundary that is also currently existing in my current work. }

    \item Why is your work important/useful/relevant?
    
    \textit{Approach: Explain the broader significance and potential applications of your research.}

    \brown{ \textbf{Answer:} Significance that I see is that the learning of new type of mathematics, and developing new techiques, as well as understanding dynamics, and how stuff like Uhlenbeck's work which is Sobolev based could be generalised to rough settings. Moreover, the problem of Coulomb gauge also has a variational formualtion and how that would be applied in the setting where everything is rough. 
    }

    \item Describe a challenging problem you've encountered in your research and the steps you took to address it.
    
    \textit{Approach: Select a specific challenge, detail your problem-solving process, and reflect on the outcome and lessons learned.}

    \brown{\textbf{Answer:} There are many challenging problems in this work. FOr example in RUC, setting up equations vs  identification, Picard iteration fixing up symmetries via enhanced Picard iteration, playing around until one figures out a Da Prato Debussche argument might do the job. Also another thing is that the result that we had, namely RUC as it stands, the idea for it was inspired by the works by Ilya, so one said it should be possible to obtain similar result. Also our initial approach that we had, we could only solve in the solution manifold, difficult to explain that was totally different, but from there we learnt that those rough addiitive functions would be useful. So I got inspirations from there to  initiate this program, and was totally outside the knowledge of Ilya and Tom btw}
    
    \item How do you plan to build upon your PhD research in this postdoctoral role?
    
    \textit{Approach: Discuss how your doctoral work serves as a foundation and outline the specific directions you'd like to explore further.}

    \brown{\textbf{Answer:} by the problems mentioned above as well as discovering new areas around my area. Expertise from Nicolas in stochastic analysis, stochastic homogenisation, regularisation by noise, energy methods}

    \item What is the most significant open problem in your field, and how might your research contribute to solving it?
    
    \textit{Approach: Identify an open problem relevant to their group and propose your approach or ideas.}

    \brown{\textbf{Answer:} Obviously there is the millinium problem on YM. I would not say that my work contributes to that in any way, I would say maybe a more feasible interesting open problem would be construction of 2D YMH. Well, Uhlenbeck and gauge fixing is very useful technique! For example, recently there is construciton of Abelian YMH via Langevin dynamic, they require Coulomb gauge. For instance, we could do something similar but by continuously applying Uhlenbeck compactness. still not clear, tho how that would work out. I mean I think Uhlenbeck compactness could be used to obtain global existence result for the Langevin dynamic.}

    \item Can you compare different approaches to handling singular SPDEs?
    
    \textit{Approach: Highlight key methods and their trade-offs.}

    \brown{\textbf{Answer:}  Yes, like paracontrolled, RS, and Flow equation. Each seems different though...}

    \item If given full autonomy, what research question would you tackle first, and why?

    \textit{Approach: Show initiative and originality.} 
    
    \brown{\textbf{Answer:} Questions that I would like to tackle first, okay if I should not care about applications immediately, it would be learning how to use Uhlenbeck for global existence of orbit under Langevin dynamic. This might require finding different kind of quantities for the equation under which Uhlenbeck's argument might apply to, or rather a singular SPDE version of it like we have. I believe with discussions with different experts in Leipzig like you Nicolas or Otto could provide new insights what this quantity might look like and how we would manage to prove the result. }
    
    

    \item What’s your dream theorem to prove in the next 3-5 years?
    
    \textit{Approach: Articulate an ambitious yet feasible goal that aligns with your expertise and field trends.}

    \brown{\textbf{Answer:} Mmmm I am not really sure what to say here, sadly having worked with quantum gauge theory made me obsessed with it, so I would say proving equivalence with Operator approach and current holonomy approach to Yang-Mills measure and and Yang-Mills measure formulation with connection forms.} 
 

    \item What’s something not in your area that you are interested in, and why?
    
    \textit{Approach: Highlight a curiosity or interdisciplinary interest, connecting it to your main research.}

    \brown{\textbf{Answer:} regularisation by noise, and I believe there is this one integral that I would love to make sense of using regularisation by noise techniques, but I currently do not have any ideas}

    \item What would you like to learn?
    
    \textit{Approach: Reflect on skills or knowledge that would enhance your research capabilities or broaden your expertise.}

    \brown{\textbf{Answer:} I would like to learn stochastic homogenisation, fast slow systems, stochastic conservation laws, stochastic fluid dynamics, and differential geometry better and maybe toplogy. The latter will enahnce my current research on Yang-Mills theory. The former I would learn to enhance my knowledge.}
    
    \item \red{(From Tom)} Interest for communities outside your exact current area?
    
    \textit{Approach: Reflect on interdisciplinary applications of your work and how it might benefit adjacent fields.}

    \brown{\textbf{Answer:} The areas I would like to enter would be for example people who do SLE and Conformal field theory.  I hope techniques from there would be benefiical to understanding EQFT but I am not sure. 
    }
    \item \red{(From Tom)} What to do if projects are not successful?
    
    \textit{Approach: Share a strategy for adapting and learning from setbacks, including pivoting to new directions or revising approaches.}

    \brown{\textbf{Answer:} I am kind of used to this as there were like three projects that I have started without any good progress. 2 from my supervisor on YM theory. 1 other one on regularisation by noise and stochastic reconstruction theorem or extension of stochastic seewing. This project started with two different people on two different seemingly similar problem, but we never like the result we could obtain for infinite dimensional equations with multiplicative noise. I guess simple answer is to learn why it did not work, maybe write down what one could try next before giving up and then set it aside. Some problems are simply too difficult.... }
    \item \red{(From Tom)} What are the most important open problems/big research challenges in your field?
    
    \textit{Approach: Identify key questions and articulate your perspective on their significance and potential solutions.}

    \brown{\textbf{Answer:} I would say critical equations for example. But easier ones could even be like 2D non-Abelian YMH measure. }
    
    \item \red{(From Tom)} Who is doing the most interesting work in your area?
    
    \textit{Approach: Demonstrate knowledge of leading researchers and their contributions, highlighting collaborations or influences.}

    \brown{\textbf{Answer:} I do not know who to be honest, but there are many people in QFT, I could say Max Gubinelli or my supervisor, but then I am biased hahaha (partially is answered in a previous question)}

    \item \red{(From Tom)} What are the big changes you expect to see in your area over the next decade?
    
    \textit{Approach: Discuss emerging trends and how they might reshape the field.}

    \brown{\textbf{Answer:} Changes? Rather than developments? Maybe there will be better understanding of path wise approaches in different aspects of probability which now seems unexpected?}

    \item \red{(From Tom)} Does your work have links to machine learning/data science?
    
    \textit{Approach: Highlight any connections, such as methodologies or potential applications, between your research and these areas.}

    \brown{\textbf{Answer:} I would say no.}

    \item \red{(From Tom)} Who would you want to invite for seminars?
    
    \textit{Approach: Suggest researchers whose work complements or enriches the group’s focus, explaining your choices.}

    \brown{\textbf{Answer:} Terry Levy, and Pawel Duch. Terry Levy to understand Yang-MIlls and discuss fruitful directions based on developments initiated by my supervisor, additive functions, now RUC etc. Pawel Duch as I have some questions that I do not understand for Flow equation, like what is the corresponding impossibility theorem for flow approach. Ways to improve regularity needed for initial conditions and coefficients.}

    \item \red{(From Tom)} Which other institutions/departments would you want to reach out to or collaborate with? How?
    
    \textit{Approach: Outline potential partnerships and the mutual benefits of such collaborations.}

    \brown{\textbf{Answer:} Maybe differential geometry people, like people who learn Calderon problems etc, inverse problems, to learn about rough inverse problems. }

     \item \blue{(For Abdul)}  We have skimmed the working version of your project that you talked about and we have seen that you have been working on the document for quite a while. Could you please explain to us what major roadblocks you encountered that resulted in the project becoming quite long?
    
    \textit{Approach: Identify specific challenges and explain how you addressed them, emphasizing persistence and learning.}

    \brown{\textbf{Answer:} The result that we want to prove with RAF, the corollary with YM at least on the square would be much easier, the version with RAF, the pathwise RUC, requires us to build machinary, like non-standard Picard iteration, regularity structures with certain subspaces, construction of model, construction of input data, all the auxiliary stuff coming from there. Furthermore, the regularity strucutres is slightly different, for example the Green function decomposition that is used in Gerencser-Hairer cannot be used here as parabolic Green funciton has exponential decay which we do not have. There is also a patching step. Finally, setting up the equation and identification is something that also makes the paper long. }
    
    \item  \blue{(For Abdul)}  Which result within your current body of work are you most proud of and why?
    
    \textit{Approach: Share a result that highlights your skills and its broader impact.}

    \brown{\textbf{Answer:} this is a subjective questions right, I would personally say Proposition  Proposition 5.7, which gives the regularity of Omega something for the term $d^*$ of white noise. For long time we expected that this requires a seperate stochastic estimate. In fact with the programm to construct the model and input data through RAF, had an additional stochastic estimate that one requires stochastic estimate for $d^*F^A$. That would be so sad, and seemingly ugly. It is also the only result that I do not have intuition why it works, for all others I have! Until I would be happy to understand thorugh different means like geometrically why it is true!}

    \item \blue{(For Abdul)}  The focus in your PhD, as we understand it, has been on Yang–Mills theory and its connections to rough paths theory and regularity structures. As you know, the focus of research in Leipzig is a bit less on mathematical physics. So which areas of overlap do you see with the topics represented here? Which people could you see yourself working with except the proposed project PI?
    
    \textit{Approach: Discuss interdisciplinary connections and potential collaborations within the department.}

    \brown{ \textbf{Answer:} Singular SPDEs, Nicolas Perkwoski and Felix Otto, Mathematical Physics and singular SPDEs also Nikolay Barashkov. Singular SPDEs also Benjamin Gess. Stochastic analysis would also include many more people. But each person how they do the things might different or so many different research etc. I could see myeslf working Nikolay Barashkov and Benjamin Gess. I could also be interested in talking with Max von Renesse and from physics Stefan Hollands.  }
    
    \item \blue{(For Abdul)}  What is the most interesting development in your area of research at the moment?
    
    \textit{Approach: Highlight cutting-edge developments and their implications for your work.}    

    \brown{\textbf{Answer:} Pushing rough analysis.}

    \item \blue{(For Abdul)} Which are the most important open questions in the field?
    
    \textit{Approach: Identify key challenges and explain their importance to advancing the field.}

    \brown{\textbf{Answer:} I would say construction of non-Abelian Yang-Mills-Higgs. I am not sure though.}
\end{enumerate}

\section{Collaboration and Teaching Questions}
\begin{enumerate}
    \item How have you collaborated with other researchers in the past, and what was the outcome?
    
    \textit{Approach: Highlight a specific collaboration, emphasizing your role and the results.}

    \brown{\textbf{Answer:} Yes, for example now with Tom and Max. This is however initiated by my supervisor. There were two collaborations on very similar topic that I initiated, one with Hannes Kern, application of stochastic reconstruction together with techniques from extension of stochastic sewing, and another with Fabian Germ on something similar for infinite dimensional SDEs with multiplicative noise. The first one was with PAM. We stopped them as we realised that it would not be a quick result project and with the first one we realised we wanted to do something slightly different in the end. Also with Fabian Germ we were close to obtaining a result but we did not like the assumptions we had to put.}
    
    \item Are you comfortable mentoring graduate students or contributing to group teaching activities?
    
    \textit{Approach: Discuss past experiences and your willingness to engage with junior researchers.}

    \brown{\textbf{Answer:} I guess  there is no really teaching in MPI. I have taught a lot. I would like to help or co-supervise students if the possibility arise. I need to practice this skill for the future and I think now is good time!}

    \item How would you handle conflicts or differing opinions within a research group?
    
    \textit{Approach: Stress your ability to listen and mediate.}

    \brown{\textbf{Answer:} I would first carefully listen to everyone's perspective and try to reach a compromise. In academia, I hope there are no personal conflicts—research conflicts should be resolved by finding reasonable compromises. In the end, we can remind ourselves that what we are doing is in the name of science, which can help put things into perspective and resolve the disagreement.}

    \item What collaborations do you have planned for this postdoc? (internal and external).
    
    \textit{Approach: Discuss potential collaborations, emphasizing how they align with the group’s work.}

    \brown{\textbf{Answer:} internal, I am very interested in the group of Felix Otto and the research conducted on variational approach to SPDEs and how that can be extended on different equations, and the current work on tree-free regularity structures, okay honestly not precisely the tree free part, but the part where one solves the equation in the solution manifold. I could try to find collaboration with the group on that work with problems inspired from geometry and quantum gauge theory. Exernal, with Ajay Chandra on understanding how one would construct GFF or $\Phi^4_2$-type measure given covariant derivative where the connection form is rough, in our case $C^{0-}$. This helps understand construction of YMH. Maybe if we push the operator approach, with Max and Ilya we can work further on developing the operator formulation for YM measure.}
\end{enumerate}

\section{Technical and Hypothetical Questions}
\begin{enumerate}
    \item How would you approach proving global existence or uniqueness for a challenging stochastic equation?
    
    \textit{Approach: Outline your strategy, referencing known methods while showing creativity in adapting them to new problems.}

    \brown{\textbf{Answer:} maybe try energy methods, a-priori bounds, techniques like absorbing initial data in some of the terms like done by Bringgman-Cao.}

    \item How would you explain the concept of rough paths to someone outside the field?
    
    \textit{Approach: Simplify the concept while retaining its mathematical essence.}

    \brown{\textbf{Answer:} maybe to someone who knows Ito-calculus, mention how the construction is not pathwise, no almost sure convergence, but rough paths does it pathwise.}
    
    \item What would you do if you encountered a significant gap in a proof during your research?
    
    \textit{Approach: Describe your process for addressing such challenges.}

    \brown{\textbf{Answer:} I feel like there are two kind of proofs. Proofs of results that are highly expected to be true or ones that is uncertain. With the ones that are highly expected to be true, I would usually go back and see where it is failing and try to weaken the statement and see what I could prove. If I can prove that maybe find whether I even needed the original result. On the way usually my experience is that changing some definitions or assumptions accordingly makes such result correct. With ones that are uncertain, yeah these are difficult. Usually try different ways or resolving it, think of similar problems in your work or literature. Ask for help from fellow PhD/ or postdocs, supervisor, mentor etc, whether they might have an idea. }

    \item Imagine you are working on a collaborative project and hit a mathematical dead end. What steps would you take?
    
    \textit{Approach: Emphasize persistence and collaboration.}

    \brown{\textbf{Answer:} for someone that went through the collabs that did not go anywhere, I would say think of useful results that got out from there, these correct results could they themselves be generalised to cover more cases? If yes, maybe do that and hope that it gets to a reasonable publication. If the results are too bad, then maybe drop it for the moment and get back to it once you see new techniques!}

   

    \item How would you approach deriving a priori estimates for a specific PDE or SPDE?
    
    \textit{Approach: Walk through your strategy step-by-step.}

    \brown{\textbf{Answer:} The ones that I know is like assuming something is small/smallness parameter absorbing on the left hand side something like that. I guess in general one could use energy estimates or something?}

    \item What numerical methods and computational tools have you used for solving PDEs or SPDEs?
    
    \textit{Approach: Describe your experience with numerical methods, highlighting the coding skills, software, or programming languages you’ve used for implementing these methods..}

    \brown{\textbf{Answer:} currently not doing numerics, but back then I have used mathematica Matlab, Python, R, but that was long time ago}

\end{enumerate}

\section{Broader Perspective Questions}
\begin{enumerate}
    \item How do you stay informed about the latest developments in stochastic analysis and related fields?
    
    \textit{Approach: Discuss your strategies for keeping up-to-date, such as attending conferences, participating in seminars, and engaging with recent literature.}

    \brown{\textbf{Answer:} Subscribing to Arxiv, analysis PDEs and probability, I mean that is not enough, but you do see what people in the area are working on. Of course, attending workshops, conferences, and seminars. Also disucssing with people about maths always recent works come up, like during lunch etc.}

    \item What do you think are the most exciting developments in stochastic analysis or PDEs in recent years?
    
    \textit{Approach: Demonstrate awareness of the field by mentioning specific breakthroughs.}

    \brown{\textbf{Answer:} I would say rough analysis, that there is this push towards splitting the probability from analysis in stochastic problems. Obviously rough paths, reguality structures, paracontrolled calculus etc etc, but I would say also stuff like proving local existence of SPDE with random intial data where the randomness is freezed and instead equipped with some suitable norms for initial data so that the equations can be studied deterministicallly, works by my supervisor collaborator, and his student as well. }

    \item Where do you see your research fitting into the broader context of mathematics or applied sciences?
    
    \textit{Approach: Show how your work has broader implications.}

    \brown{\textbf{Answer:} I hope that my work has implications for mathematical physics, understanding the building blocks of the universe, I am totally aware that what I am currently doing does not directly influence any understanding. But in the long run, I hope these little bits commulate to a better understanding. Also singular SPDEs arise outside physics, I guess in biology as well (works by Mayorcas, Martini). }

    \item How do you see stochastic analysis impacting other fields of mathematics or science?
    
    \textit{Approach: Highlight interdisciplinary connections.}

    \brown{\textbf{Answer:} stuff like Feyman-Kac type formulas help understand PDEs etc. Some bounds in differential geometry can be proved with some Brownian motion, I guess the works by Karl-Theodor Sturm, Eva Kopfer?}

   

\end{enumerate}

\section{Career and Motivation Questions}
\begin{enumerate}
    \item What are your long-term research goals, and how does this postdoctoral position fit into them?
    
    \textit{Approach: Outline your research aspirations and explain how the position at MPI Leipzig will facilitate achieving them.}

    \brown{\textbf{Answer:} I see this postdoc as a great opportunity to learn about other fields within, say rough analysis, like rough paths to fast-slow systems, stochastic homogenisation, as well as other problems in singular SPDEs like energy methods. These new fields will equip me with better understanding of variaty of problems to be able to establish a bigger network of people I could work with and establish at the same time an independent programme. This will help me in the end to apply for fellowships etc afterwards. In the long run it helps me to apply for permannent positions as well. }
    
 

      \item \blue{(For Abdul)} Why are you applying to Leipzig? What attracts you here and why is now a good time to be here?
    
    \textit{Approach: Highlight specific aspects of Leipzig and its research environment that align with your goals.}

    \brown{\textbf{Answer:} Firstly, I am aware on how big the research groups are whose research are I am interested in to learn and evolve in. Outside you two and your PhD and postdocs, I would mention the PIs Ben Gess and Nikolay Barashkov. I expect that the research activity will be really good through listening to the group members and me sharing my work. I am hoping to many new directions for me. Markus Tempelmayr really recommended me to come here! Furthermore, there is this huge CRC grant with Berlin and other universities in Germany which provides networking oppurtinities with different universities as well. I have also wanted to work with Nicolas Perkwoski and be at the same time in Leipzig means that I also work with the groups here!  }  
    

    \item How do you plan to measure your success during your postdoc?
    
    \textit{Approach: Highlight clear goals and metrics.}

    \brown{\textbf{Answer:} objective answer is through publications and outreach of research through conferences/workshops etc. But I believe, as publications is as important, I also appreciate what I have learnt in the journey regardless of the number of publications. My success I also measure to understand how old problems I have tried are still untacklable with new experts for example. This is also an assuring thing to know.}
   

    \item What is the purpose of this postdoc for you?
    
    \textit{Approach: Reflect on how this position aligns with your research and career trajectory.}

    \brown{\textbf{Answer:} somehow mentioned before, as a step to learn from Nicolas Perkwoski, the community in Leipzig as well as having a broad networking with people so that I have a group of people I collaborate with and have somehow formed my own research paths for the future.}
\end{enumerate}


\section{Personal Questions}
\begin{enumerate}
    \item \red{(From Tom)} Where do you see yourself in 5 years?
    
    \textit{Approach: Reflect on your career trajectory and articulate specific goals, such as positions, research achievements, or collaborations.}

    \brown{\textbf{Answer:} I see myself, hopefully, having a permanent position somewhere in Europe, likely in Germany, as I have a different kind of attraction towards Germany haha. I hope I have some kind of fellowship somewhere and applying for permanent positions. }

    \item \red{(From Tom)} What do you want to be known for in 5 years' time?
    
    \textit{Approach: Highlight the unique contributions or areas of expertise you aim to establish in your field.}

    \brown{\textbf{Answer:} Operator approach for 2D Yang-Mills theory, but that might be ambitious. Maybe contributions to pushing pathwise approaches to surprising areas. }

    



    \item \blue{(For Abdul)} If we were to offer you a position, when would you be able to commit and when would you be able to start?
    
    \textit{Approach: Provide clear timelines, showing enthusiasm and flexibility.}

    \brown{\textbf{Answer:} I think usually one week is enough, I need to mentally prepare for the decision I would take.}
\end{enumerate}




\section{Soft Skills and Non-Research Skills}
\begin{enumerate}
    \item What organizational experience do you have?
    
    \textit{Approach: Highlight instances where you demonstrated leadership, coordination, or event planning.}

    \brown{\textbf{Answer:} I organised two reading groups, one on Malliavin calculus, and another one on rough paths and machine learning, and long time ago, one-day workshops for calculus students.}

    \item Tell us about a time you experienced a setback.
    
    \textit{Approach: Share a specific example, focusing on how you overcame the challenge and what you learned.}

    \brown{\textbf{Answer:} the first one was when my first project on defining lasso fields through pathwise tehcniques would be dropped, it taught me a lot about academia. One could have super interesting questions but simply not have the tools/knowledge to solve  it, in fact, after spending so much time, you feel like you wasted so much time. I seen it as a lesson. That happesn and we do not know the future, maybe simply learn how to move on and know maybe for the future to connect it back once new methodologies arise. }

    \item What are your strengths and weaknesses?
    
    \textit{Approach: Be honest, highlighting how you manage weaknesses and leverage strengths.}

    \brown{\textbf{Answer:} I would say my strength is pushing, and maybe in some cases brute forcing a lot of techniques, I will be honest the Picard iteration and setting up the equation was kind of obtained that way, that is also a weakness, I think in some cases I initially miss the intuition, but develop it later on. I am learning how to develop intuition, and I like proofs that I understand why they are true regardless of the mathematical proofs. Like why and how? That usually takes time. I'd say my weakness is not seeing the future. Like, some people can see very soon whether a project is useful or not.}
    
    \item How would your advisor describe you?
    
    \textit{Approach: Reflect on feedback or comments you’ve received from mentors.}

    \brown{\textbf{Answer:} I think he would describe as a strong candidate, he is aware that I can get good contributions to research, especially since this project was difficult to get it moving and my contributions made it move to what it currently is. }

    
    \item What are your back-up plans?
    
    \textit{Approach: Outline alternative career paths or research directions you’re prepared to pursue.}

    \brown{\textbf{Answer:} my backup plans to apply to more positions, I do really like doing research. There is more people whose research I would like to understand. I would hope I could get a position and learn from others. I am also planning to learn a little bit machine learning techniques, I am also developing side interest in AI in music, was talking to a friend about it, maybe work in such industry does not sound too bad, plus I would still do mathematics, but not singular SPDEs unfortunately haha}
    
     \item \red{(From Tom)} Administration (or "service") is part of all academic life. Do you have experience with admin roles in the past? What roles do you see yourself in here? What outside of research and teaching would you bring to the department? (again more for permanent posts)
    
    \textit{Approach: Reflect on past administrative roles and how they prepared you for potential responsibilities.}

    \brown{\textbf{Answer:} Not many to be honest, like organising reading groups, and gathering people for it.}
    \subsection{Funding}
\begin{enumerate}
    

    \item \red{(From Tom)} Experience applying for funding?
    
    \textit{Approach: Share examples of successful applications or insights gained from the process.}

    \brown{\textbf{Answer:} only for travel funding.}

    \item \red{(From Tom)} Do you think you would like to lead a research group in the near future? If so, what size? (X \# PhDs, Y \# postdocs.) (similar to career trajectory questions above)
    
    \textit{Approach: Articulate your vision for a research group, focusing on structure, goals, and collaboration.}

    \brown{\textbf{Answer:} I am too early to have thought about this, but the program for 2D YM to connect three different notions.}
\end{enumerate}

\end{enumerate}

\section{Questions for the Interviewers}
\subsection{Research Opportunities and Group Dynamics}
\begin{enumerate}
    \item How does the group encourage collaboration among researchers with diverse expertise?

 %   \item Are there opportunities to co-supervise graduate students or participate in joint projects?

    \item  How is feedback typically provided on ongoing research projects within the group?

    \item \red{(NICE)} Are there group seminars, like the seminar at FU Berlin where group memebers talk about their work and additional external members?

    \item \red{(NICE)} How the work as postdoc will be? Like meeting with mentor? Who will I be meeting? 
\end{enumerate}

\subsection{Career Development}
\begin{enumerate}
    \item Are there formal mentoring or career development programs for postdocs?

    \item \red{(NICE)} What are typical next steps for postdocs in your group after completing their term?
\end{enumerate}

\subsection{Resources and Support}
\begin{enumerate}
%    \item Are there resources available for developing computational tools or accessing specific datasets?

    \item How much flexibility do postdocs have in proposing and pursuing independent projects?

    \item \red{(NICE)} Will there be funding for research visits and conferences? (How much?)

    \item \red{(NICE)} Does MPI support/fund me to follow German classes? 
\end{enumerate}

\subsection{Institutional Involvement}
\begin{enumerate}
    \item Are there opportunities to participate in interdisciplinary collaborations with other MPI groups or institutions?

    \item \red{(NICE)} Are there any opportunities to organise the seminar,  workshops or summer schools?

    \item Does MPI Leipzig provide access to workshops, conferences, or training programs for skill development?
\end{enumerate}

\subsection{Group Vision and Expectations}
\begin{enumerate}
    \item \red{(NICE)} What are the group’s priorities or focus areas for the next few years?

    \item How is the balance maintained between theoretical work and applications within the group?

    \item \red{(NICE)} When is the expected start date?
\end{enumerate}


\end{document}